
General notes

NVS

Note that this receiver uses UTC as the reference timescale to report time stamps.

This receiver reports a measurement time for observation a few hundred ms prior to the upcoming
second.

Trimble

 

Internals

One aspect to keep straight is that three timescales are used in the software:
PC time
UTC
GPS time

PC time is used to match TIC and GNSS observations. The time stamps recorded in measurement files do
not have a fractional seconds part. The latency of the various signals (eg GPS messages 
for a second are always output after the beginning of the second) and their logging by the host PC 
means that this is no ambiguity during normal operation.

UTC time is used for CGGTTS generation.

GPS time is used for various calculations and for RINEX observations.

\subsection{Application.cpp}

Matched measurements are stored in a vector whose index corresponds to UTC time-of-day.

\subsection{CGGTTS.cpp}


\section{Adding support for a new receiver}

\subsection{Conventions}

A counter/timer measurement must be started by REF and stopped by GPS.
There is an option in gpscv.conf to reverse the sign of this.

The sawtooth correction is ADDED to the counter/timer measurement.

\subsection{Configuring the receiver}

It may be necessary to turn off tracking of non-GPS GNSS systems.

It may be desirable to configure the receiver's refernce time scale 
\subsection{Clocks and pseudoranges}

When developing for a new receiver, interpreting the usually terse documentation can require
guesswork. The main problem is to make sure that you can establish the relationship between the raw 
pseudoranges and the output 1 pps. The receiver measurements will likely be reported with respect to the 
receiver clock, which will necessarily have an offset with respect to the refernec timescale.


It can be very helpful to have another, already-supported receiver on the same antenna. The pseudo ranges reported 
by this receiver can be used to identify any mysterious offsets in the pseudoranges 

\subsection{Sawtooth correction}

For CGGTTS output, the sawtooth correction may not have much effect at the 780s averaging time implicit in a CGGTTS track.

\subsection{Diagnostics}

Extra diagnostic files can be produced via command line options.
The option \cc{--timing-diagnostics} produces a text file \cc{timing.dat}. This text file has four columns:
	\begin{description*}
	\item[1] timestamp, in seconds since beginning of UTC day
	\item[2] TIC measurement, in seconds
	\item[3] 1 pps sawtooth correction, in seconds, to be added to the TIC measurement
	\item[4] receiver time offset, in seconds
	\end{description*}

The option \cc{--sv-diagnostics} produces a test file \cc{SVn.dat} for each GNSS satellite. This text file has
XXX columns
	\begin{description}
	\item[1] interpolated pseudo range
	\item[2] raw pseudo range
	\end{description}
	
\subsection{Debugging and validation}

It can be useful to look at how well the receiver recovers GPS time - this is easily done by
using the option --disable-tic. The sawtooth-corrected TIC measurement is then set to zero.

REFSYS values noisy at the hundreds of ns level may indicate an off by one error in assigned time stamps.

