
General notes

NVS

Note that this receiver uses UTC as the reference timescale to report time stamps.

This receiver reports a measurement time for observation a few hundred ms prior to the upcoming
second.

Trimble

 

\subsection{Time scales}

Three timescales are used in the software:
\begin{enumerate}
\item PC time
\item UTC
\item GPS time
\end{enumerate}

PC time must be used to match TIC and GNSS measurements. The time stamps recorded in measurement files do
not have a fractional seconds part. The latency of the various signals (eg GPS messages 
for a second are always output after the beginning of the second) and their logging by the host PC 
means that there is no ambiguity during normal operation ie when the PC is properly synchronized.

UTC time is used for CGGTTS generation.

GPS time is used for various calculations and for RINEX observations.

\subsection{Application.cpp}

\subsection{CGGTTS.cpp}


\subsection{Adding support for a new receiver}

A new receiver is added by deriving a new class from \cc{Receiver}. This class has one virtual member
function that must be redefined \cc{Receiver::readLog}.

Edit the Makefile to add the new receiver class files to the dependencies.

\cc{Application.cpp} must be edited to create an instance of the new receiver, when configured.
This is done in \cc{Application::loadConfig()}.

In terms of the configuration file \cc{gpscv.conf}, a receiver can be defined by three entries:
\cc{receiver::model}, \cc{receiver::manufacturer} and \cc{receiver::version}.

\subsection{Identifying suitable receivers for time-transfer}

The basic requirement is that the receiver be capable of outputting raw code measurements.
The reference time scale for these measurements needs to be known, as well as any varying offset between
the reference time scale and the code measurements. It is also necessary to know any varying offset
between the reference time scale and the output 1 pps. The 1 pps sawtooth correction is not necessary,
at least if only CGGTTS output is desired.

For standalone CGGTTS generation, the receiver must also amake available ephemeris information, including
satellite orbits, the ionosphere model and ...


\subsection{Configuring the receiver}

It may be necessary to turn off tracking of non-GPS GNSS systems.

It may be desirable to configure the receiver's reference time scale 
\subsection{Clocks and pseudoranges}

When developing for a new receiver, interpreting the usually terse documentation can require
guesswork. The main problem is to make sure that you can establish the relationship between the raw 
pseudoranges and the output 1 pps. The receiver measurements will likely be reported with respect to the 
receiver clock, which will necessarily have an offset with respect to the refernce timescale.

It can be very helpful to have another, already-supported receiver on the same antenna. The pseudo ranges reported 
by this receiver can be used to identify any mysterious offsets in the pseudoranges 

\subsection{Sawtooth correction}

For CGGTTS output, the sawtooth correction may not have much effect at the 780s averaging
time implicit in a CGGTTS track. GNSS receivers appear to be designed so that the period of the sawtooth is
short, presumably so that it can be averaged out quickly.

The second to which the sawtooth correction is applied may not be explicitly stated and may apply to the previous
or next output 1 pps.

\subsection{Diagnostics}

Extra diagnostic files can be produced via command line options.
The option \cc{--timing-diagnostics} produces a text file \cc{timing.dat}. This text file has four columns:
	\begin{description*}
	\item[1] timestamp, in seconds since beginning of UTC day
	\item[2] TIC measurement, in seconds
	\item[3] 1 pps sawtooth correction, in seconds, to be added to the TIC measurement
	\item[4] receiver time offset, in seconds
	\end{description*}

The meaning of the ``receiver time offset'' depends on the receiver. In the case of the 

The option \cc{--sv-diagnostics} produces a test file \cc{SVn.dat} for each GNSS satellite. This text file has
XXX columns
	\begin{description}
	\item[1] interpolated pseudo range
	\item[2] raw pseudo range
	\end{description}
	
\subsection{Debugging and validation}

It can be useful to look at how well the receiver recovers GPS time - this is easily done by
using the option --disable-tic. The sawtooth-corrected TIC measurement is then set to zero.

REFSYS values noisy at the hundreds of ns level may indicate an off by one error in assigned time stamps.

