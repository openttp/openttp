\section{sysmonitor.pl \label{ssysmonitor}}

\cc{sysmonitor.pl} monitors the system status and provides notification of alarm conditions via 
files written to a specified directory, and via calling the alarm delivery system. In particular, 
\cc{lcdmonitor} reads this directory to pick up current alarms.

Some of the conditions currently monitored include:
\begin{itemize}
\item TIC logging running (via it's status file)
\item reference oscillator logging running
\item reference is locked
\item reference has lost power (PRS10 only)
\item GPS logging running
\item GPS receiver is tracking sufficient satellites
\item RAID status (where RAID is used)
\item NTP reference clocks are healthy
\end{itemize}

The run time for an alarm must integrate up to the configured threshold before an alarm is issued. 
Similarly, the run time for a clearing alarm must integrate to zero before a clear is issued.

\subsection{usage}
\cc{sysmonitor.pl} is normally started by the init system, for example by \cc{systemd} on Debian.

To run \cc{sysmonitor.pl} on the command line, use:
\begin{lstlisting}
sysmonitor.pl [OPTION]
\end{lstlisting}
The command line options are:
\begin{description*}
	\item[-c \textless file\textgreater]	use the specified configuration file 
	\item[-d]	run in debugging mode
	\item[-h]	print help and exit
	\item[-v]	print version information and exit
\end{description*}
To manually run \cc{okcounterd}, you may need to disable the system service
and kill any running \cc{okcounterd} process.

\subsection{configuration file}

The configuration file uses the format described in \ref{sConfigFileFormat}.\\

{\bfseries alarm path}\\
This defines the  directory to which alarm notifications are written.\\
\textit{Example:}
\begin{lstlisting}
alarm path = /usr/local/log/alarms
\end{lstlisting}
{\bfseries alarm threshold}\\
This defines the  threshold at which alarms are raised. The units are seconds.\\
\textit{Example:}
\begin{lstlisting}
alarm threshold = 60
\end{lstlisting}
{\bfseries alerter queue}\\
Alarms can be delivered by other methods using \cc{alerter}. This entry defines the queue used by \cc{alerter}.
\textit{Example:}
\begin{lstlisting}
alerter queue = /usr/local/log/alert.log
\end{lstlisting}
{\bfseries gpscv account}\\
This defines the account used for GPSCV processing (and implicitly, the location of \cc{gpscv.conf}).\\
\textit{Example:}
\begin{lstlisting}
gpscv account = cvgps
\end{lstlisting}
{\bfseries log file}\\
This defines the file used for logging of sysmonitor's operation and alarm events.\\
\textit{Example:}
\begin{lstlisting}
log file = /usr/local/log/sysmonitor.log
\end{lstlisting}
{\bfseries ntp account}\\
This defines the account used for NTP-related logging and processing.\\
\textit{Example:}
\begin{lstlisting}
ntp account = ntp-admin
\end{lstlisting}
{\bfseries ntpd refclocks}\\
This specifies a list of sections, each of which defines an \cc{ntpd} refclock that is to be monitored.\\
\textit{Example:}
\begin{lstlisting}
ntpd refclocks = PPS,NMEA
\end{lstlisting}

An \cc{ntpd} refclock section looks like:\\
\begin{lstlisting}
[NMEA]
refid = 127.127.20.0
name = NMEA
\end{lstlisting}


\subsection{log file}
\cc{sysmonitor.pl} creates a log file. The default file is \cc{/usr/local/log/sysmonitor.log}