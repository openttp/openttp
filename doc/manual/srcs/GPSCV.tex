
This chapter describes software related to GPSCV time-transfer.


\section{Software overview}

Time-transfer files are produced by \cc{mktimetx} from the GNSS receiver logs and counter/timer measurements.
The time-transfer files are in  RINEX observation  and CGGTTS format. 
Currently, CGGTTS-format files are only produced for GPS. 
Figure \ref{f:GPSCVProcessing} illustrates the processing chain TODO. 

The OpenTTP software suite is catalogued in Table \ref{t:OTTPSoftware}
\begin{table}
\begin{tabular}{l|l|l}
	& program & \\ 
	\hline
GPSCV processing  &  & \\
	& \hyperlink{h:mktimetx}{mktimetx} & core program\\
	& \hyperlink{h:runmktimetx}{runmktimetx.pl} & \\
	& \hyperlink{h:mkcggtts}{mkcggts.py} & \\
	\hline
TIC logging & & \\
	& \hyperlink{h:hp5313xlog}{hp5313xlog.pl} &\\
	& \hyperlink{h:cnt9xlog}{cnt9xlog.py} &\\
	& \hyperlink{h:okxemlog}{okxemlog.pl} & \\
	& \hyperlink{h:prs10log}{prs10log.pl} & \\
	& \hyperlink{h:ticclog}{ticclog.py} & \\
	\hline
GNSS receiver logging & & \\
	&	\hyperlink{h:jnslog}{jnslog.pl} & Javad\\
	& \hyperlink{h:nvslog}{nv08log.pl} & NVS\\
	& \hyperlink{h:restlog}{restlog.pl} & Trimble Resolution T\\
	& \hyperlink{h:ublox9log}{ublox9log.py} & ublox\\
	& \hyperlink{h:ubloxlog}{ubloxlog.pl} & ublox\\
GNSS receiver utilities & & \\
	& \hyperlink{h:jnsextract}{jnsextract.pl} & \\
	& \hyperlink{h:nv08extract}{nv08extract.pl} & \\
	& \hyperlink{h:nv08info}{nv08info.pl} & \\
	& \hyperlink{h:restextract}{restextract.pl} & \\
	& \hyperlink{h:restinfo}{restinfo.pl} & \\
	& \hyperlink{h:ubloxextract}{ubloxextract.py} & \\
	& \hyperlink{h:ubloxmkdev}{ubloxmkdev.py} & \\
	\hline
Analysis tools & & \\
	& \hyperlink{h:cggttsqc}{cggttsqc.py} & \\
	& \hyperlink{h:cmpcggtts}{cmpcggtts.py} & \\
	& \hyperlink{h:editcggtts}{editcggtts.py} & \\
	& \hyperlink{h:editrnxnav}{editrnxnav.py} & \\
	& \hyperlink{h:editrnxobs}{editrnxobs.py} & \\
	& \hyperlink{h:ticqc}{ticqc.py} & \\
Miscellaneous & & \\
  & \hyperlink{h:log1Wtemp}{log1Wtemp.pl} & \\ 
	\hline
\end{tabular}
\caption{GPSCV software overview \label{t:OTTPSoftware} }
\end{table}

\section{crontab \label{ss:crontab}}

Automatic logging, processing and archival of data is co-ordinated via the user \cc{cvgps}' \cc{crontab}.

A minimal \cc{crontab} for the user \cc{cvgps} looks like this:
\begin{lstlisting}
# Check that all logging is running every 5 minutes
*/5 * * * * /usr/local/bin/kickstart.pl # See ~/etc/kickstart.conf

# Run the processing of the data at 00:15
15 0 * * * nice $HOME/bin/runmktimetx.pl >/dev/null 2>&1 

# Give the processing some time to complete, then zip the files at 00:45
45 0 * * * nice $HOME/bin/gziplogs.pl >/dev/null 2>&1

# Check the status of the system once a day, just before the day rollover
56 23 * * * $HOME/bin/checkstatus.pl >$HOME/lockStatusCheck/status.dat
\end{lstlisting}
showing the three essential processes of logging, processing and archival of data.

The active crontab can be examined with the command \cc{crontab -l}. 
A default \cc{crontab} is saved in \cc{/home/cvgps/etc/crontab}. 

\section{Configuration file format \label{sConfigFileFormat}}

Configuration files use a common format and are plain text files, designed to be easily edited via a command-line
editor because in many applications, only shell access to the system will be available.

The file is usually divided into sections, with section names delimited by square brackets [ ]. Entries in each section
are of the form:
\begin{lstlisting}
key = value
\end{lstlisting}
For example,
\begin{lstlisting}
[Receiver]
manufacturer = Trimble
model = Resolution T
\end{lstlisting}
defines a section \cc{Receiver} and the two keys: \cc{manufacturer} and \cc{model}. 

The notation \cc{Section::Key} is used to fully specify keys. For example,
\cc{Receiver::model} and \cc{Receiver::manufacturer} specify the two keys above.

Keys and section names are not case-sensitive. In particular, the python and Perl libraries
which provide functions for reading the configuration files convert keys and section names to lower case.
The case of key values is preserved, since this may be signficant eg path names.

Leading and trailing whitespace is removed from keys, key values and section names.

Comments begin with a `\#'. 

Some keys define a list of sections. 
For example, the comma-separated list of values for \cc{CGGTTS::outputs} 
\begin{lstlisting}
[CGGTTS]
outputs = c1-code,p1-code,p2-code
\end{lstlisting}
defines three sections: \cc{c1-code}, \cc{p1-code}, and \cc{p2-code}.
This is a bodge which would be more elegantly addressed using an extensible format like XML, 
but it has proven to be sufficient for our needs.

\subsection{Paths} \label{s:ConfigFilePaths}

Paths to files  specified in a configuration file are constructed with the following 
precedence:
\begin{enumerate}
\item If the path begins with a leading slash, then it is interpreted as an absolute path
\item If a non-absolute path is specified, it is interpreted as being relative to 
	the users's home directory (or where the configuration file allows specification of a different root path eg in \hyperlink{h:rootpath}{gpscv.conf}, relative to that root path) 
\item Otherwise, the default path is used.
\end{enumerate}
Most software in OpenTTP follows these conventions.

\section{Data file formats \label{s:DataFileFormat}}

\subsection{GPS receiver}

This text file records messages from the GPS receiver. 
The native format of the messages can be a mix of ASCII and binary messages.
Binary messages are hexadecimal-encoded for saving in the log file. 
Some logging scripts record ancillary information, such as commands sent to the receiver. 
Comments in the log file are allowed, prefaced by a `\#' character. 
The `@' character is used to tag lines containing special information that needs to be parsed by
the processing software.

Messages are successive lines of the form:
\begin{lstlisting}
<message_id> <time_stamp> <message>
\end{lstlisting}
\textit{Example:}
\begin{lstlisting}
TO 00:00:02 cdfbc75a9a8c353fc5
\end{lstlisting}

Hex encoding of binary messages results in much larger files but these compress to a size not much larger
than the original binary data.

The exception to this is the Septentrio logging script, which uses the native SBF format.
Some logging scripts have the option of logging data in the receiver's native format, to facilitate
use of other tools.

\subsection{Time-interval counter \label{s:TICformat}}

This text file records the difference between GNSS receiver and the Reference Oscillator 1 pps,
measured each second. The convention is that the Reference oscillator provides the start trigger.
Entries are successive lines of the form:
\begin{lstlisting}
<time_of_day> <time_difference>
\end{lstlisting}
where the time difference is in seconds.\\
\textit{Example:}
\begin{lstlisting}
00:00:04  +4.0821E-006 
\end{lstlisting}

\section{gpscv.conf - the core configuration file \label{sgpscvconf} }

A single configuration file, \cc{gpscv.conf}, provides configuration information to most of the
OpenTTP software. 
\cc{gpscv.conf} is used by \cc{mktimetx}, receiver logging scripts, TIC logging scripts,receiver utilities and so on.

It uses the format described in \ref{sConfigFileFormat}.

\begin{table}[h]
\begin{tabular}{l|p{10cm}}
Section & Key \\ \hline
\hyperlink{h:antenna}{Antenna} & antenna number, antenna type, 
				delta H, delta N, delta E, frame,
				marker name, marker number,
				X, Y, Z\\ \hline
\hyperlink{h:cggtts}{CGGTTS} & BIPM cal id, comments, create, 
         ephemeris, ephemeris file, ephemeris path,
         internal delay, lab id, maximum DSG, minimum elevation,
         minimum track length, naming convention, outputs, reference,
         receiver id, revision date, version,
         code, constellation, path
         \\ \hline
\hyperlink{h:delays}{Delays}  & antenna cable, reference cable 
         \\ \hline
\hyperlink{h:counter}{Counter} & file extension, GPIB address, header generator, lock file,
         logger, logger options, okxem channel, port
				\\ \hline
\hyperlink{h:misc}{Misc}    & gzip
				\\ \hline
\hyperlink{h:paths}{Paths} & CGGTTS, counter data, processing log, receiver data, RINEX, tmp
				\\ \hline
\hyperlink{h:receiver}{Receiver} & configuration, elevation mask, logger, logger options, 
				 manufacturer, model, observations, 
         port, pps offset, synchronization, pps synchronization delay,
         status file, timeout, version
				\\ \hline
\hyperlink{h:reference}{Reference} & file extension, logging interval, log path, log status, oscillator, power flag, status file
        \\ \hline
\hyperlink{h:rinex}{RINEX}  & agency, create, observer, version
				\\ \hline
\end{tabular}
\caption{Summary of \cc{gpscv.conf} entries}
\end{table}

\subsection{[Antenna] section}

\hypertarget{h:antenna}{}
{\bfseries antenna number}\\
This appears as ANTNUM in the RINEX header.\\
\textit{Example:}
\begin{lstlisting}
antenna number=A567456
\end{lstlisting}

{\bfseries antenna type}\\
This appears as ANTTYPE in the RINEX header.\\
\textit{Example:}
\begin{lstlisting}
antenna type=Ashtec
\end{lstlisting}

{\bfseries delta H}\\
This appears as DELTA H in the RINEX header.\\
\textit{Example:}
\begin{lstlisting}
delta H=0.0
\end{lstlisting}

{\bfseries delta E}\\
This appears as DELTA E in the RINEX header.\\
\textit{Example:}
\begin{lstlisting}
delta E=0.0
\end{lstlisting}

{\bfseries delta N}\\
This appears as DELTA N in the RINEX header.\\
\textit{Example:}
\begin{lstlisting}
delta N=0.0
\end{lstlisting}

{\bfseries frame}\\
This appears as FRAME in the CGGTTS header.\\
\textit{Example:}
\begin{lstlisting}
frame= ITRF2010
\end{lstlisting}

{\bfseries marker name}\\
This appears as MARKER NAME in the RINEX header.\\
\textit{Example:}
\begin{lstlisting}
marker name=
\end{lstlisting}

{\bfseries marker number}\\
This appears as MARKER NUMBER in the RINEX header.\\
\textit{Example:}
\begin{lstlisting}
marker number=
\end{lstlisting}

{\bfseries X}\\
This appears as X in the CGGTTS header and APPROX POSITION XYZ in the RINEXheader.\\
\textit{Example:}
\begin{lstlisting}
X=+4567890.123
\end{lstlisting}

{\bfseries Y}\\
This appears as Y in the CGGTTS header and APPROX POSITION XYZ in the RINEXheader.\\
\textit{Example:}
\begin{lstlisting}
Y=+2345678.90
\end{lstlisting}

{\bfseries Z}\\
This appears as Z in the CGGTTS header and APPROX POSITION XYZ in the RINEXheader.\\
\textit{Example:}
\begin{lstlisting}
Z=-1234567.890 
\end{lstlisting}


\subsection{[CGGTTS] section }

\hypertarget{h:cggtts}{Entries} in this section control the format and content of CGGTTS files and filtering applied to CGGTTS tracks.

{\bfseries BIPM cal id}\\
This defines CAL\_ID for the internal delay, as used in v2E CGGTTS headers.\\
\textit{Example:}
\begin{lstlisting}
BIPM cal id=none
\end{lstlisting}

{\bfseries comments}\\
This defines COMMENTS in the CGGTTS header.\\
\textit{Example:}
\begin{lstlisting}
comments=none
\end{lstlisting}

{\bfseries create}\\
This defines whether or not CGGTTS files will be generated.\\
\textit{Example:}
\begin{lstlisting}
create=yes
\end{lstlisting}

{\bfseries ephemeris}\\
This defines whether to use the receiver-provided ephemeris or a user-provided ephemeris (via a RINEX navigation file).
If a user-provided ephemeris is specified then \cc{ephemeris path} and \cc{ephemeris file} 
must also be specified.\\
\textit{Example:}
\begin{lstlisting}
ephemeris=receiver
\end{lstlisting}

{\bfseries ephemeris file}\\
This specifies a pattern for user-provided RINEX navigation files.
Currently, only patterns of the form \cc{XXXXddd0.yyn} are recognized.\\
\textit{Example:}
\begin{lstlisting}
ephemeris file=SYDNddd0.yyn
\end{lstlisting}

{\bfseries ephemeris path}\\
This specifies the path to user-provided RINEX navigation files.\\
\textit{Example:}
\begin{lstlisting}
ephemeris path=igsproducts
\end{lstlisting}

{\bfseries internal delay}\\
This defines INT DLY in the CGGTTS header. The units are ns.\\
\textit{Example:}
\begin{lstlisting}
INT DLY=0.0
\end{lstlisting}

{\bfseries lab id}\\
This defines the two-character lab code used for creating BIPM-style file names, as per the V2E specification.\\
\textit{Example:}
\begin{lstlisting}
lab id=AU
\end{lstlisting}

{\bfseries maximum DSG}\\
CGGTTS tracks with DSG lower than this will be filtered out. 
The units are ns.\\
\textit{Example:}
\begin{lstlisting}
maximum DSG = 10.0
\end{lstlisting}

{\bfseries minimum elevation}\\
CGGTTS tracks lower than this will be filtered out. 
The units are degrees.\\
\textit{Example:}
\begin{lstlisting}
minimum elevation = 10
\end{lstlisting}

{\bfseries minimum track length}\\
CGGTTS tracks shorter than this will be filtered out. Tracks meeting the criterion are not necessarily contiguous.
The units are seconds.\\
\textit{Example:}
\begin{lstlisting}
minimum track length = 390
\end{lstlisting}

{\bfseries naming convention}\\
Defines the CGGTTS file naming convention. Valid options are `plain' (MJD.cctf) and `BIPM'.
The \cc{lab id} and \cc{receiver id} should be defined in conjunction with BIPM-style filenames.\\
\textit{Example:}
\begin{lstlisting}
naming convention = BIPM
\end{lstlisting}

{\bfseries outputs}\\
This defines a list of sections which in turn define the desired CGGTTS outputs.\\
\textit{Example:}
\begin{lstlisting}
outputs=CGGTTS-GPS-C1,CGGTTS-GPS-P1,CGGTTS-GPS-P2
\end{lstlisting}

{\bfseries reference}\\
This defines REF in the CGGTTS header.\\
\textit{Example:}
\begin{lstlisting}
reference=UTC(XXX)
\end{lstlisting}

{\bfseries receiver id}\\
This defines the two-character receiver code used for creating BIPM-style file names, 
as per the V2E specification.\\
\textit{Example:}
\begin{lstlisting}
receiver is=01
\end{lstlisting}

{\bfseries revision date}\\
This defines REV DATE in the CGGTTS header. It must be in the format YYYY-MM-DD.\\
\textit{Example:}
\begin{lstlisting}
revision date = 2015-12-31
\end{lstlisting}

{\bfseries version}\\
This defines the version of CGGTTS output.Valid versions are v1 and v2E. 
The \cc{lab id} and \cc{receiver id} should be defined in conjunction with v2E ouput\\
\textit{Example:}
\begin{lstlisting}
version = v2E
\end{lstlisting}

\subsubsection{CGGTTS output sections}

Multiple CGGTTS outputs can be defined, allowing for different constellation and signal combinations.
An example of a CGGTTS output section is as follows:
\begin{lstlisting}
[CGGTTS-GPS-C1]
constellation=GPS
code=C1
path=cggtts
BIPM cal id = none
internal delay = 11.0
\end{lstlisting}

The new entries for a CGGTTS output section are:\\
{\bfseries code}\\
This defines the GNSS signal code. Valid values are C1,P1 and P2.\\
\textit{Example:}
\begin{lstlisting}
code=C1
\end{lstlisting}

{\bfseries constellation}\\
This defines the GNSS constellation. Only GPS is supported currently.\\
\textit{Example:}
\begin{lstlisting}
constellation=GPS
\end{lstlisting}

{\bfseries path}\\
This defines the directory in which output files are placed.\\
\textit{Example:}
\begin{lstlisting}
path=cggtts
\end{lstlisting}

\subsection{[Counter] section}

\hypertarget{h:counter}{}

{\bfseries file extension}\\
This defines the extension used for time interval measurement files.
The default is `tic'.\\
\textit{Example:}
\begin{lstlisting}
file extension=tic
\end{lstlisting}

{\bfseries GPIB address}\\
For GPIB devices, the GPIB address must be defined.
\textit{Example:}
\begin{lstlisting}
GPIB address=3
\end{lstlisting}

{\bfseries header generator}\\
A header for the log file can be optionally added to the log file, using the output
of a user provided script. Output should be to \cc{stdout}.
Each line will have a ``\#'' automatically prepended to it.\\
\textit{Example:}
\begin{lstlisting}
header generator=bin/myticheader.pl
\end{lstlisting}

{\bfseries lock file}\\
This defines the device lock file, used to prevent multiple instances of the logger
from being started.\\
\textit{Example:}
\begin{lstlisting}
lockfile = okxem.gpscv.lock
\end{lstlisting}

{\bfseries logger}\\
This defines the counter logging script.\\
\textit{Example:}
\begin{lstlisting}
logger=okxemlog.pl
\end{lstlisting}

{\bfseries logger options}\\
This defines options passed to the counter logging script.\\
\textit{Example:}
\begin{lstlisting}
logger options=
\end{lstlisting}

{\bfseries okxem channel}\\
The OpenTTP counter is multi-channel so the channel to use (1-6) must be specified.\\
\textit{Example:}
\begin{lstlisting}
okxem channel=3
\end{lstlisting}

{\bfseries port}\\
This defines the port used to communicate the counter. It's value depends on the type of counter. 
For the XEM6001, it's a Unix socket. For serial devices, it's a device name like
\cc{/dev/ttyUSB0}.\\
\textit{Example:}
\begin{lstlisting}
# this is the port used by okcounterd
port = 21577 
\end{lstlisting}

\subsection{[Misc section}

\hypertarget{h:misc}{}

{\bfseries gzip}\\
Defines the compression/decompression program used in conjunction with counter and receiver log files.\\
\textit{Example:}
\begin{lstlisting}
gzip=/bin/gzip 
\end{lstlisting}

\subsection{[Delays] section}

\hypertarget{h:delays}{}

{\bfseries antenna cable}\\
This is ANT DLY as used in the CGGTTS header. Units are ns.\\
\textit{Example:}
\begin{lstlisting}
antenna cable=0.0
\end{lstlisting}

{\bfseries reference cable}\\
This is REF DLY as used in the CGGTTS header. Units are ns.\\
\textit{Example:}
\begin{lstlisting}
reference cable=0.0
\end{lstlisting}

\subsection{[Paths] section}

\hypertarget{h:paths}{Paths} are relative to the user's home directory, unless prefaced with a `/', in which case
they are interpreted as absolute paths.

{\bfseries CGGTTS}\\
Defines the default directory used for CGGTTS files.\\
\textit{Example:}
\begin{lstlisting}
CGGTTS=cggtts
\end{lstlisting}

{\bfseries counter data}\\
Defines the directory used for TIC data files.\\
\textit{Example:}
\begin{lstlisting}
counter data=
\end{lstlisting}

{\bfseries processing log}\\
Defines the directory where the \cc{mktimetx} processing log is written.\\
\textit{Example:}
\begin{lstlisting}
processing log=logs
\end{lstlisting}

{\bfseries receiver data}\\
Defines the directory used for GNSS receiver raw data files.\\
\textit{Example:}
\begin{lstlisting}
receiver data=raw
\end{lstlisting}

{\bfseries RINEX}\\
Defines the directory used for RINEX files.\\
\textit{Example:}
\begin{lstlisting}
RINEX=rinex
\end{lstlisting}

{\bfseries tmp}\\
Defines the directory used for intermediate and debugging files.\\
\textit{Example:}
\begin{lstlisting}
tmp=tmp
\end{lstlisting}


\subsection{[Receiver] section \label{sgcreceiver}}

\hypertarget{h:receiver}{}

{\bfseries configuration}\\
\\
\textit{Example:}
\begin{lstlisting}
configuration = etc/rx.conf
\end{lstlisting}

{\bfseries elevation mask}\\
This sets an elevation mask for tracking os satellites - below this elevation, satellites
are ignored. The units are degrees. This may not be implemented for all receivers.\\
\textit{Example:}
\begin{lstlisting}
elevation mask = 0
\end{lstlisting}

{\bfseries logger}\\
This is the script used to configure and log messages from the GNSS receiver.\\
\textit{Example:}
\begin{lstlisting}
logger = jnslog.pl
\end{lstlisting}

{\bfseries logger options}\\
These are options passed to the receiver logging script.\\
\textit{Example:}
\begin{lstlisting}
logger options =
\end{lstlisting}

{\bfseries manufacturer}\\
This defines the manufacturer of the receiver. Together with the model and version, 
this sets how data from the receiver is processed. For a list of supported receivers
see XX.\\
\textit{Example:}
\begin{lstlisting}
manufacturer = Javad
\end{lstlisting}

{\bfseries model}\\
This is the receiver model.For a list of supported receivers
see XX.\\
\textit{Example:}
\begin{lstlisting}
model = HE_GGD
\end{lstlisting}

{\bfseries observations}\\
This is a list of GNSS systems tracked by the receiver. Only GPS is supported
at present. Although the receiver model defines the possible observations,
it may be configured to track only one GNSS system, so this entry specifies which
one is being tracked.\\
\textit{Example:}
\begin{lstlisting}
observations = GPS
\end{lstlisting}

{\bfseries port}\\
This is the serial port used for communication with the receiver.\\
\textit{Example:}
\begin{lstlisting}
port = /dev/ttyS0
\end{lstlisting}

{\bfseries pps offset}\\
This is an offset programmed into the GNSS receiver. Its purpose is to ensure that the counter
triggers correctly, particularly HP5313x counters, which will only trigger every two seconds if the 
start trigger slips slightly behind the stop trigger. It is only useful though if the reference 
(which by convention is Start) has comparable long-term stability with GPS eg a GPSDO or Cs beam standard.
The units are ns.\\
\textit{Example:}
\begin{lstlisting}
pps offset = 3500
\end{lstlisting}

{\bfseries pps synchronization}\\
This is a Javad-specific option. The logging script will force a synchronization of the receiver's
internal time scale with the input 1 pps.\\
\textit{Example:}
\begin{lstlisting}
pps synchronization = no
\end{lstlisting}

{\bfseries pps synchronization delay}\\
This is a Javad-specific option. Synchronization of the receiver's
internal time scale with the input 1 pps is delayed for this time after reset of the receiver.
The units are seconds.\\
\textit{Example:}
\begin{lstlisting}
pps synchronization delay = 300
\end{lstlisting}

{\bfseries status file}\\
\\
\textit{Example:}
\begin{lstlisting}
status file =
\end{lstlisting}

{\bfseries timeout}\\
The logging script will time out and exit if no messages are received for this period.\\
\textit{Example:}
\begin{lstlisting}
timeout = 60
\end{lstlisting}

{\bfseries version}\\
This can be used to identify the firmware version in use, for example.\\
\textit{Example:}
\begin{lstlisting}
version = 2.6.1
\end{lstlisting}

\subsection{[Reference] section}

\hypertarget{h:reference}{}

{\bfseries file extension}\\
This defines the extension used for Reference status logs.\\
\textit{Example:}
\begin{lstlisting}
file extension= .rb
\end{lstlisting}

{\bfseries logging interval}\\
This defines the interval between status file updates. The units are seconds.\\
\textit{Example:}
\begin{lstlisting}
logging interval=60
\end{lstlisting}

{\bfseries log path}\\
This defines where status logs are written to.\\
\textit{Example:}
\begin{lstlisting}
log path=raw
\end{lstlisting}

{\bfseries log status}\\
This enables status logging of the Reference.\\
\textit{Example:}
\begin{lstlisting}
log status=yes
\end{lstlisting}

{\bfseries oscillator}\\
This identifies the installed oscillator, so that device-specific handling can be implemented.\\
\textit{Example:}
\begin{lstlisting}
oscillator=PRs10
\end{lstlisting}

{\bfseries power flag}\\
This defines the file used to flag that the Reference has lost power, and needs rephasing.
Currently, this only has meaning for the PRS10. It is used ntpd to disable the refclock
corresponding to the PRs10's 1 pps.\\
\textit{Example:}
\begin{lstlisting}
power flag=logs/prs10.pwr
\end{lstlisting}

{\bfseries status file}\\
In the case of the PRS10, this consists of the six status bytes and sixteen ADC values.\\
\textit{Example:}
\begin{lstlisting}
status file=logs/prs10.status
\end{lstlisting}


\subsection{[RINEX] section}

\hypertarget{h:rinex}{Entries} in this section control the format and content of RINEX files.

{\bfseries agency}\\
This specifies the value of the AGENCY field which appears in RINEX observation file headers.\\
\textit{Example:}
\begin{lstlisting}
agency=MY AGENCY
\end{lstlisting}

{\bfseries create}\\
This defines whether or not RINEX files will be generated.\\
\textit{Example:}
\begin{lstlisting}
create = yes
\end{lstlisting}

{\bfseries observer}\\
This specifies the value of the OBSERVER field which appears in RINEX observation file headers.
If the observer is specified as `user' then the environment variable USER is used.\\
\textit{Example:}
\begin{lstlisting}
observer=user
\end{lstlisting}

{\bfseries version}\\
This  specifies the version of the RINEX output. Valid versions are 2 and 3.\\
\textit{Example:}
\begin{lstlisting}
version=2
\end{lstlisting}





General notes

NVS

Note that this receiver uses UTC as the reference timescale to report time stamps.

This receiver reports a measurement time for observation a few hundred ms prior to the upcoming
second.

Trimble

 

Internals

One aspect to keep straight is that three timescales are used in the software:
PC time
UTC
GPS time

PC time is used to match TIC and GNSS observations. The time stamps recorded in measurement files do
not have a fractional seconds part. The latency of the various signals (eg GPS messages 
for a second are always output after the beginning of the second) and their logging by the host PC 
means that this is no ambiguity during normal operation.

UTC time is used for CGGTTS generation.

GPS time is used for various calculations and for RINEX observations.

\subsection{Application.cpp}

Matched measurements are stored in a vector whose index corresponds to UTC time-of-day.

\subsection{CGGTTS.cpp}


\section{Adding support for a new receiver}

\subsection{Conventions}

A counter/timer measurement must be started by REF and stopped by GPS.
There is an option in gpscv.conf to reverse the sign of this.

The sawtooth correction is ADDED to the counter/timer measurement.

\subsection{Configuring the receiver}

It may be necessary to turn off tracking of non-GPS GNSS systems.

It may be desirable to configure the receiver's refernce time scale 
\subsection{Clocks and pseudoranges}

When developing for a new receiver, interpreting the usually terse documentation can require
guesswork. The main problem is to make sure that you can establish the relationship between the raw 
pseudoranges and the output 1 pps. The receiver measurements will likely be reported with respect to the 
receiver clock, which will necessarily have an offset with respect to the refernec timescale.


It can be very helpful to have another, already-supported receiver on the same antenna. The pseudo ranges reported 
by this receiver can be used to identify any mysterious offsets in the pseudoranges 

\subsection{Sawtooth correction}

For CGGTTS output, the sawtooth correction may not have much effect at the 780s averaging time implicit in a CGGTTS track.

\subsection{Diagnostics}

Extra diagnostic files can be produced via command line options.
The option \cc{--timing-diagnostics} produces a text file \cc{timing.dat}. This text file has four columns:
	\begin{description*}
	\item[1] timestamp, in seconds since beginning of UTC day
	\item[2] TIC measurement, in seconds
	\item[3] 1 pps sawtooth correction, in seconds, to be added to the TIC measurement
	\item[4] receiver time offset, in seconds
	\end{description*}

The option \cc{--sv-diagnostics} produces a test file \cc{SVn.dat} for each GNSS satellite. This text file has
XXX columns
	\begin{description}
	\item[1] interpolated pseudo range
	\item[2] raw pseudo range
	\end{description}
	
\subsection{Debugging and validation}

It can be useful to look at how well the receiver recovers GPS time - this is easily done by
using the option --disable-tic. The sawtooth-corrected TIC measurement is then set to zero.

REFSYS values noisy at the hundreds of ns level may indicate an off by one error in assigned time stamps.




\cc{runmktimetx.pl} provides a convenient way to process multiple days of data and to run any missed processing.

	
\subsection{usage}
\cc{runmktimetx.pl} is normally run as a \cc{cron} job.

To run on the command line, use
\begin{lstlisting}
runmktimetx.pl [OPTION] \textellipsis [Start MJD  [Stop MJD]]
\end{lstlisting}
If an MJD or MJD range is not specified, the previous day is processed.

The options are
\begin{description*}
	\item[-d]	run in debugging mode
	\item[-h]	print help and exit
	\item[-x] run missed processing
	\item[-v]	print version information and exit
\end{description*}

\subsection{configuration file}
\cc{runmktimetx.pl} uses \cc{gpscv.conf}.

\subsection{log file}
\cc{runmktimetx.pl} doesn't produce a log file.

\subsection{mkcggtts.py}

\hypertarget{h:mkcggtts}{}

\cc{mkcggtts.py} is used to generate CGGTTS files from RINEX navigation and observation files using
a third part tool. Currently, the only tool supported is \cc{r2cggtts}.

\subsubsection{usage}

\begin{lstlisting}[mathescape=true]
mkcggtts.py [OPTION] $\ldots$ mjd [mjd $\ldots$]
\end{lstlisting}
The command line options are:
\begin{description*}
	\item[-{}-help,-h]	print the help information and exit
	\item[-{}-debug,-d]	print debugging information to stderr
	\item[-{}-leapsecs \textless{n}\textgreater]	set the number of leap seconds
	\item[-{}-previousmjd]	when no MJD (or one MJD) is given, MJD-1 is added to the MJDs to be processed
	\item[-{}-version,-v] show the version information and exit
\end{description*}

The leap second count in \cc{paramCGGTTS} is determined from the RINEX navigation file if possible.
Otherwise, it can be specified manually via  the \cc{--leapsecs} option.

The \cc{--previousMJD} option is intended to be used in automated daily processing using \cc{r2cggtts}, 
to reprocess data for any missed track at the end of the day.

\subsubsection{configuration file}

See the sample!

\subsubsection{examples}

None yet!


\section{cnt9xlog.py}

\hypertarget{h:cnt9xlog}{Pendulum} CNT-90 and CNT-91 counters are supported via USBTMC (in particular, the Python \cc{usbtmc} module).

\subsection{usage}

\begin{lstlisting}[mathescape=true]
cnt9xlog.py [option] $\ldots$ 
\end{lstlisting}

The command line options are:
\begin{description*}
	\item[-c \textless file\textgreater] use the specified configuration file
	\item[-d]	run in debugging mode
	\item[-h]	print help and exit
	\item[-v]	print version information and exit
\end{description*}

\subsection{configuration file}

There is an optional file \cc{cnt9x.cmds} which lists SCPI commands used to configure the counter.
This file overrides the default configuration:
\begin{lstlisting}
:SENS:TINT:AUTO OFF
:SENS:FUNC 'TINT 1,2'
:INP1:COUP DC                   # coupling DC
:INP2:COUP DC
:INP1:IMP 50                    # impedance 50 ohms
:INP2:IMP 50
:INP1:LEVEL 1.0
:INP2:LEVEL 1.0
:INP1:SLOPE POS
:INP2:SLOPE POS
\end{lstlisting}





\section{hp5313xlog.pl}

\hypertarget{h:hp5313xlog}{HP and Agilent} 53131 and 53132 counters are supported, in combination
with the IOTech GPIB to RS232 converter.

\subsection{usage}

\begin{lstlisting}[mathescape=true]
hp5313xlog.pl [option] $\ldots$ 
\end{lstlisting}

The command line options are:
\begin{description*}
	\item[-c \textless file\textgreater] use the specified configuration file
	\item[-d]	run in debugging mode
	\item[-h]	print help and exit
	\item[-v]	print version information and exit
\end{description*}

\subsection{configuration file}

There is a file \cc{hp5313x.cmds} which lists the SCPI commands used to configure the counter.
For example:
\begin{lstlisting}
:FUNC 'TINT 1,2'                # time interval
:SENS:EVEN1:LEVEL:ABS 1.0       # trigger level 1 volt
:SENS:EVEN2:LEVEL:ABS 1.0       #
:SENS:EVEN1:SLOP POS            # trigger on positive slope
:SENS:EVEN2:SLOP POS
:INP1:ATT 1                     # input attenuation x1
:INP2:ATT 1
:INP1:COUP DC                   # coupling DC
:INP2:COUP DC
:INP1:IMP 50                    # impedance 50 ohms
:INP2:IMP 50
\end{lstlisting}

It has the following specific configuration file entries in \cc{gpscv.conf}:
\begin{itemize}
	\item \hyperlink{h:counter_configuration}{counter:configuration}
	\item \hyperlink{h:counter_gpib_address}{counter:gpib address}
	\item \hyperlink{h:counter_gpib_converter}{counter:gpib converter}
\end{itemize}



\section{okxemlog.pl}

\hypertarget{h:okxemlog}{}
The OpenTTP reference platform includes a multi-channel TIC and this script communicates with the daemon
\cc{okcounterd} via a local TCP/IP socket. The script will exit if no data are returned for more than two minutes.

\subsection{usage}

\begin{lstlisting}[mathescape=true]
okxemlog.pl [option] $\ldots$ 
\end{lstlisting}

The command line options are:
\begin{description*}
	\item[-c \textless file\textgreater] use the specified configuration file
	\item[-d]	run in debugging mode
	\item[-h]	print help and exit
	\item[-v]	print version information and exit
\end{description*}

It has the following specific configuration file entries in \cc{gpscv.conf}:
\begin{itemize}
	\item \hyperlink{h:counter_okxem_channel}{counter::okxem channel}.
\end{itemize}


\section{prs10log.pl}

\hypertarget{h:prs10log}{}

The Stanford PRS10 rubidium standard has a 1 pps input port that is typically used
to lock the PRS10 to a GNSS receiver. The PRS10 timetags each input 1 pps with respect to its own 1 pps and can report these measurements. It can thus be also be used as a time-interval counter. 
In this application, the lock to the input 1 pps is disabled 
(ie the PRS10 is left free-running),
and the 1 pps measurements are used for time-transfer.

\subsection{usage}

\begin{lstlisting}[mathescape=true]
prs10log.pl [option] $\ldots$ 
\end{lstlisting}

The command line options are:
\begin{description*}
	\item[-c \textless file\textgreater] use the specified configuration file
	\item[-d]	run in debugging mode
	\item[-h]	print help and exit
	\item[-v]	print version information and exit
\end{description*}

The PRS10 is also used as the system time reference so it has
entries in \cc{gpscv.conf} associated with this:
\begin{itemize}
	\item \hyperlink{h:reference_log_status}{reference::log status}
	\item \hyperlink{h:reference_logging_interval}{reference::logging interval}
	\item \hyperlink{h:reference_log_path}{reference::log path}
	\item \hyperlink{h:reference_file_extension}{reference::file extension}
	\item \hyperlink{h:reference_power_flag}{reference::power flag}
	\item \hyperlink{h:reference_status_file}{reference::status file}
\end{itemize}


\section{ticclog.py}
\hypertarget{h:ticclog}{}

\cc{ticclog.py} is used to log data from a TAPR Timestamping/TIme Interval Counter (TICC).
See \cc{https://www.tapr.org/kits\_ticc.html} for more information about this counter.

The command line options are:
\begin{description*}
 \item[-h, -{}-help] print help and exit
 \item[-{}-config \textless file\textgreater, -c \textless file\textgreater] use the specified configuration file
 \item[-{}-debug, -d]           run in debugging mode
 \item[-{}-settings, -s]        print the counter settings
  \item[-{}-version, -v]        print version information and exit
\end{description*}

There are no specific configuration file entries for this script.


\section{Javad/Topcon receivers}

\subsection{jnslog.pl}
\hypertarget{h:jnslog}{}
There is a file \cc{receiver.conf} which lists the commands used to configure the receiver.

Configuration commands in \cc{gpscv.conf} specific to this receiver are:

JNS receivers have a number of specific configuration entries in \cc{gpscv.conf}:
\begin{itemize}
\item \hyperlink{h:configuration}{receiver::configuration}
\item \hyperlink{h:pps_synchronization}{receiver::pps synchronization}
\item \hyperlink{h:pps_synchronization_delay}{receiver::pps synchronization delay}
\item \hyperlink{h:rinex_l1l2}{paths::rinex l1l2}
\end{itemize}

\subsection{jnsextract.pl}
\hypertarget{h:jnsextract}{}

\subsection{runrinexobstc.pl}


\section{NVS NV08C receivers}

\subsection{nv08log.pl}
\hypertarget{h:nvslog}{}

The NVS receiver is entirely configured using the script.

\subsection{nv08extract.pl}

\hypertarget{h:nv08extract}{}

\subsection{nv08info.pl}

\hypertarget{h:nv08info}{}


\section{Septentrio receivers}

A typical processing chain for the Septentrio receiver looks like:

\begin{enumerate}
\item \cc{plrxlog.py}    to log the receiver
\item \cc{runsbf2rnx.py} to generate RINEX files from SBF
\item \cc{mkcggtts.py}   to generate CGGTTS files from RINEX
\end{enumerate}

Currently, the processing chain relies on the tools provided by Septentrio and r2cggtts.
Open source replacements for these are currently under development.
OpenTTP now provides \cc{sbf2rnx} as a replacement for sbf2rin, but this is limited to GPS at present.

\subsection{plrxlog.py}
\hypertarget{h:plrxlog}{}

\cc{plrxlog.py} is used to configure and log Septentrio receivers.
Unlike the other receiver logging scripts, it logs using the receiver's native binary data format, rather
than the custom OpenTTP format.
It has mainly been used with mosaicT receivers.

It doesn't produce a log file.

\subsubsection{usage}

\begin{lstlisting}[mathescape=true]
plrxlog.py [option] $\ldots$ configurationFile
\end{lstlisting}

The command line options are:
\begin{description*}
	\item[--config \textless{file}\textgreater, -c \textless{file}\textgreater] use the specified configuration file
	\item[--debug, -d]	run in debugging mode
	\item[--reset, -r]  reset receiver and exit
	\item[--help, -h]	print help and exit
	\item[--version, -v]	print version information and exit
\end{description*}

A reset command issues:
\begin{lstlisting}
exeResetReceiver,Hard,PVTData+SatData
\end{lstlisting}

\subsubsection{configuration}

The file \cc{receiver.conf} lists any custom commands used to configure the receiver.
These are written in Septentrio XXX format, and passed verbatim to the receiver.
For example,
\begin{lstlisting}
lstInternalFile,Identification
SetPPSparameters,sec1,Low2High,0,RxClock,60
SetTimeSyncSource,EventA
SetSignalTracking,+GPSL1PY+BDSB2A+GALE5
\end{lstlisting}

The default configuration for the mosaicT enables SBF output and the message set needed for RINEX generation and the SatVisibility message.

\subsection{mosaicmkdev.py}

\cc{mosaicmkdev.py} is used to create a symbolic link for the mosaic-T receiver.
Currently, it identifies a receiver by the serial number of the USB hub
that the various available USB devices in the mosaic-T (development kit) are accessed through.
It is typically used with a \cc{udev} rule, an example of which is included in the OpenTTP distribution.

The default configuration file is \cc{/usr/local/etc/mosaicmkdev.conf} which looks like:
\begin{lstlisting}
[main]
receivers = mos01

[mos01]
serial number = 3612929

# Symlinks to be made
ttyACM0 = mos01ACM0
ttyACM1 = mos01ACM1
\end{lstlisting}

The \cc{receivers} option is a comma-separated list of section names, as per the usual convention.
Each section defines the setup for a particular receiver.

The serial number is currently matched through the \cc{udev} `root device' property `ID\_SERIAL\_SHORT'.

The symlinks to be made in \cc{/dev} are defined by the Linux device names (\cc{ttyACM0}, \cc{ttyACM1} only at present).

Startup errors reported via \cc{udev} can be examined using
\begin{lstlisting}
systemctl status systemd-udevd
\end{lstlisting}

\subsection{runsbf2rnx.py}

\cc{runsbf2rnx.py} is a wrapper for \cc{sbf2rin}, provided by Septentrio.

\subsubsection{usage}

\begin{lstlisting}[mathescape=true]
runsbf2rnx.py [OPTION] $\ldots$ [mjd [mjd $\ldots$]]
\end{lstlisting}
The command line options are:
\begin{description*}
	\item[-{}-help,-h]	print the help information and exit
	\item[-{}-config \textless{file}\textgreater, -c \textless{file}\textgreater] use the specified configuration file 
	\item[-{}-debug,-d]	print debugging information to stderr
	\item[-{}-version,-v] show the version information and exit
\end{description*}

\subsubsection{configuration}

{\bfseries [Main] section}\\

{\bfseries exec}\\
This specifies the path to the executable \cc{sbf2rin}.
The default is \cc{/usr/local/bin/sbf2rin}.

{\bfseries sbf station name}\\

{\bfseries [Receiver] section}\\

{\bfseries file extension}\\

{\bfseries [RINEX] section}\\

{\bfseries version}\\
This option specifies the RINEX version for the output, via the option to \cc{sbf2rin}

\textit{Example:}
\begin{lstlisting}
version = 3 
\end{lstlisting}

{\bfseries name format}\\

{\bfseries obs directory}\\

{\bfseries nav directory}\\

{\bfseries obs sta}\\

{\bfseries nav sta}\\

{\bfseries create nav file}\\

{\bfseries exclusions}\\
This option specifies GNSS to be excluded from the generated RINEX files via the `-x' option to \cc{sbf2rin}.
The GNSS are specified using the same syntax as \cc{sbf2rin} uses.\\
\textit{Example:}
\begin{lstlisting}
exclusions = ISJ 
\end{lstlisting}

{\bfseries fix header}\\
This option specifies whether or not the RINEX header should be edited.

{\bfseries header fixes}\\
This option specifies the file to use when fixing the RINEX header. The file specifies replacements for lines in the
RINEX header, and needs to be in RINEX format.

{\bfseries fix satellite count}\\
Some older versions of \cc{sbf2rin} incorrectly report the number of satellites in the RINEX observation file header
when GNSS are excluded via the `-x' option to \cc{sbf2rin}. The offending line in the RINEX header is removed, since
it is optional.\\
\textit{Example:}
\begin{lstlisting}
fix satellite count = yes 
\end{lstlisting}

{\bfseries [Paths] section}\\

{\bfseries receiver data}\\

{\bfseries tmp}\\

\subsection{sbf2rinbatch.py}

\cc{sbf2rinbatch.py} is used for batch processing of SBF files via Septentrio's tool \cc{sbf2rin}.

\subsection{mksephourly.py}

\cc{mksephourly.conf} is used for hourly generation of REFSYS from CGGTTS files. 
It is intended for `live' monitoring of time-transfer.
The script calls \cc{runsbf2rnx.py} to generate RINEX, and then \cc{mkcggtts.py} to generate CGGTTS.
A new file named MJD.dat is created each day, and updated by rewriting it each time \cc{mksephourly.py} is called.

The format of this file is:
\begin{lstlisting}
MJD TOD (in seconds) REFSYS (mean) number of tracks
\end{lstlisting}

The command line options are:
\begin{description*}
	\item[--config \textless{file}\textgreater, -c \textless{file}\textgreater] use the specified configuration file
	\item[--debug, -d]	run in debugging mode
	\item[--help, -h]	print help and exit
	\item[--version, -v]	print version information and exit
\end{description*}

The default configuration file is \cc{~/etc/mksephourly.conf} which looks like:
\begin{lstlisting}
[Main]
runsbf2rnx conf = etc/hourly.runsbf2rnx.conf
cggtts source = CGGTTS-GPS-P3
mkcggtts conf = etc/hourly.mkcggtts.conf
summary path = hourly_tt/summary
\end{lstlisting}

{\bfseries runsbf2rnx conf}\\
This defines the configuration file to be used by \cc{runsbf2rnx.py}.

{\bfseries mkcggtts conf}\\
This defines the configuration file to be used by \cc{mkcggtts.py}.

{\bfseries cggtts source}\\
This defines the section in the configuration file used by \cc{mkcggtts} for CGGTTS file generation.
It is used to locate and identify the CGGTTS files that REFSYS is extracted from.

{\bfseries summary path }\\
This defines where summary files are written to.

\subsection{sbf2rnx}
Experimental!


\section{Trimble Resolution T receivers}

\subsection{restlog.pl}

\hypertarget{h:restlog}{}

The Resolution T receiver is entirely configured using the script.

\subsection{restextract.pl}

\hypertarget{h:restextract}{}

\subsection{restinfo.pl}

\hypertarget{h:restinfo}{}

\subsection{restconfig.pl}

\hypertarget{h:restconfig}{}

\subsection{restplayer.pl}


\section{ublox receivers}

\subsection{ublox9log.py}

\hypertarget{h:ublox9log}{}

This \cc{python3} script is used with ublox series 9 receivers like the ZED-F9P and ZED-F9T.
The following messages are enabled for 1 Hz output:
\begin{itemize}
	\item RXM-RAWX 
	\item TIM-TP
	\item NAV-SAT (not logged)
	\item NAV-TIMEUTC
	\item NAV-CLOCK
\end{itemize}

The command line options are:
\begin{description*}
 \item[-h, -{}-help] print help and exit
 \item[-{}-config \textless file\textgreater, -c \textless file\textgreater] use the specified configuration file
 \item[-{}-debug, -d]           run in debugging mode
 \item[-{}-reset, -r]           reset the receiver
  \item[-{}-version, -v]        print version information and exit
\end{description*}

A status file is written once per minute, containing the SV identifiers of  tracked satellites with
code and time synchronization flags set.

\subsection{ubloxlog.pl}

\hypertarget{h:ubloxlog}{}

The ublox receiver is entirely configured using the script.

\subsection{ubloxextract.py}

\hypertarget{h:ubloxextract}{}

\subsection{ubloxmkdev.py}

\hypertarget{h:ubloxmkdev}{}



\section{Miscellaneous tools}

\subsection{cggttsqc.py}

\hypertarget{h:cggttsqc}{}

\cc{cggttsqc.py} performs various checks on CGGTTS files.
Running it on a single CGGTTS file with no options will produce output like:

\begin{tabular}{lrrrrrr}
File         & Tracks  & Short & min SV & max SV   &  DSG  &  elv \\
57401.cctf   &   706   &  71   &   3    &  11      & 0     &  0   \\
\end{tabular}

where:\\
`Tracks' is the total number of tracks\\
`Short'  is the number of tracks with length less than a specified threshold (default 780 s)\\
`min SV' is the minimum number of SVs visible\\
`max SV' is the maximum number of SVs visible \\
`DSG'    is the number of tracks with DSG above a specified threshold (default 20 ns)\\
`elv'    is the number of tracks with elevation below a specified threshold (default 10 degrees)\\

\subsubsection{usage}

\begin{lstlisting}[mathescape=true]
cggttsqc.py [OPTION] $\ldots$ file [file $ldots$]
\end{lstlisting}
The command line options are:
\begin{description*}
	\item[-{}-help,-h]	print the help information and exit
	\item[-{}-debug,-d]	print debugging information to stderr
	\item[-{}-nowarn]   suppress warnings (eg about missing files, bad formatting in CGGTTS files)
	\item[-{}-DSG DSG]  set the upper limit for acceptable DSG. The units are ns.
	\item[-{}-elevation ELEVATION] set the lower limit for acceptable elevation. The units are degrees.
	\item[-{}-tracklength  ELEVATION] set the lower limit for acceptable track length. The units are seconds.
	\item[-{}-checkheader] show when significant fields in the header change, for example the delays.
	\item[-{}-sequence, -s] interpret the input filenames as a sequence.
	\item[-{}-version,-v] show the version information and exit
\end{description*}
Mutiple files can be checked and a sequence can be specified with two file names and the \cc{--sequence} option.
For example:
\begin{lstlisting}
cggttsqc.py -s --DSG 5 GZAA0157.834 GZAA0157.876
\end{lstlisting}
will report on all files between MJDs 57834 and 57876, indicating in the DSG field the number of tracks with DSG greater
than 5 ns.


\subsection{cmpcggtts.py}

\cc{cmpcggtts.py} matches tracks in paired CGGTTS files, and outputs 

\subsubsection{usage}

\begin{lstlisting}[mathescape=true]
cmpcggtts.py [OPTION] $\ldots$ refDir calDir firstMJD lastMJD
\end{lstlisting}
The command line options are:
\begin{description*}
 \item[-{}-help]            show this help message and exit
 \item[-{}-starttime STARTTIME] start time of day HH:MM:SS for first MJD (default 0)
 \item[-{}-stoptime STOPTIME]   stop time of day HH:MM:SS for last MJD (default 23:59:00)
 \item[-{}-elevationmask ELEVATIONMASK] elevation mask (in degrees, default 0.0)
 \item[-{}-mintracklength MINTRACKLENGTH] minimum track length (in s, default 750)
 \item[-{}-maxdsg MAXDSG]       maximum DSG (in ns, default 20.0)
 \item[-{}-matchephemeris ]     match on ephemeris (default no)
  \item[-{}-cv]                  compare in common view (default)
  \item[-{}-aiv]                 compare in all-in-view
  \item[-{}-acceptdelays]        accept the delays (no prompts in delay calibration mode)
  \item[-{}-delaycal]            delay calibration mode
  \item[-{}-timetransfer]        time-transfer mode (default)
  \item[-{}-ionosphere]          use the ionosphere in delay calibration mode (default = not used)
  \item[-{}-refprefix REFPREFIX] file prefix for reference receiver (default = MJD)
  \item[-{}-calprefix CALPREFIX] file prefix for calibration receiver (default = MJD)
  \item[-{}-refext REFEXT]       file extension for reference receiver (default = cctf)
  \item[-{}-calext CALEXT]       file extension for calibration receiver (default = cctf)
  \item[-{}-debug, -d]           debug (to stderr)
  \item[-{}-nowarn]              suppress warnings
  \item[-{}-quiet]               suppress all output to the terminal
  \item[-{}-keepall]             keep tracks after the end of the day
  \item[-{}-version, -v]         show version and exit
\end{description*}


\subsection{editcggtts.py}

\hypertarget{h:editcggtts}{}

\cc{editcggtts.py} is used to edit CGGTTS navigation files. 
The header checksum and the modification time are updated after editing, 

\subsubsection{usage}

\begin{lstlisting}[mathescape=true]
editcggtts.py [option] $\ldots$ filename [filename $\ldots$]
\end{lstlisting}
The command line options are:
\begin{description*}
	\item[-{}-help,-h]	print the help information and exit
	\item[-{}-debug,-d]	print debugging information to stderr
	\item[-{}-comments \textless{value}\textgreater] set comment
  \item[-{}-output,-o \textless{value}\textgreater] output to file/directory 
  \item[-{}-tmp]    output is to file(s) with .tmp added to name
  \item[-{}-replace, -r]  replace the edited file(s)
  \item[-{}-nosequence]  do not interpret (two) input file names as a sequence
  \item[-{}-nowarn]  suppress warnings
	\item[--version,-v] show the version information and exit
\end{description*}

The \cc{--output}, \cc{--tmp} and \cc{--replace} options are mutually exclusive. 
If none of these options are used, output will be to \cc{stdout}.

If two filenames are given without the \cc{--nosequence} option, they will be
interpreted as specifying a sequence of files to be processed.


\subsection{editrnxnav.py}

\hypertarget{h:editrnxnav}{}

\cc{editrnxnav.py} is used to edit RINEX navigation files. Currently, it doesn`t do much
but it will eventually do much more. 

\subsubsection{usage}

\begin{lstlisting}[mathescape=true]
cggttsqc.py [OPTION] $\ldots$ file 
\end{lstlisting}
The command line options are:
\begin{description*}
	\item[-{}-help,-h]	print the help information and exit
	\item[-{}-debug,-d]	print debugging information to stderr
	\item[-{}-output OUTPUT, -o OUTPUT] output to file/directory
  \item[-{}-replace, -r]         replace edited file
  \item[-{}-ura URA, -u URA]     remove entries with URA greater than this
	\item[--version,-v] show the version information and exit
\end{description*}



\subsection{editrnxobs.py}

\hypertarget{h:editrnxobs}{}

\cc{editrnxobs.py} is used to edit RINEX observation files. 

\subsubsection{usage}

\begin{lstlisting}[mathescape=true]
editrnxobs.py [option] $\ldots$ file [file $\ldots$] 
\end{lstlisting}
The command line options are:
\begin{description*}
	\item[-{}-help, -h]            print the help information and exit
  \item[-{}-debug, -d]           print debugging information to stderr
  \item[-{}-output,-o \textless{file}\textgreater]  output to file/directory
  \item[-{}-keep, -k]            keep intermediate files
  \item[-{}-replace, -r]         replace edited file
  \item[-{}-system \textless{system}\textgreater ]      gnss system (BeiDou,Galileo,GPS,GLONASS)
  \item[-{}-obstype \textless{obstype}\textgreater]     observation type (C2I,L2I,...)
  \item[-{}-fixms ]              fix ms ambiguities (ref RINEX file required)
  \item[-{}-fixmissing ]         add observations missing at the beginning of the day
  \item[-{}-sequence, -s ]       interpret input files as a sequence
  \item[-{}-refrinex \textless{file}\textgreater ]  reference RINEX file for fixing ms ambiguities (name of first file if multiple input files are specified)
  \item[-{}-version, -v ]        show the version information and exit
\end{description*}



\subsection{fetchigs.py}

\hypertarget{h:fetchigs}{}

\cc{fetchigs.py} uses the Python library \cc{urllib} to download GNSS products and station observation
files from IGS data centres. You will need a configuration file that sets up downloads from at least one IGS data
centre. The sample configuration file should be sufficient for most uses.

\subsubsection{usage}

\begin{lstlisting}[mathescape=true]
fetchigs.py [OPTION] $\ldots$ <start> [<stop>]
\end{lstlisting}

The \cc{start} and \cc{stop} times can be in the format:
\begin{description*}
\item[MJD] Modified Julian Date
\item[yyyy-doy] year and day of year (1 \textellipsis)
\item[yyyy-mm-dd] year, month (1-12) and day (1 \textellipsis)
\end{description*}

The command line options are:
\begin{description*}
	\item[-{}-help, -h] print help and exit
	\item[-{}-config \textless CONFIG \textgreater] , -c use this configuration file
	\item[-{}-debug, -d]           print debugging output to stderr
	\item[-{}-outputdir \textless DIR \textgreater] set the output directory
	\item[-{}-ephemeris]           get broadcast ephemeris
	\item[-{}-clocks]              get clock products (.clk)
	\item[-{}-orbits]              get orbit products (.sp3)
	\item[-{}-rapid]               fetch rapid products
	\item[-{}-final]               fetch final products
	\item[-{}-centre \textless CENTRE \textgreater]       set data centres
	\item[-{}-listcentres, -l]     list available data centres
	\item[-{}-observations]        get station observations (only mixed observations for V3)
	\item[-{}-statid \textless STATID \textgreater]      station identifier (eg V3 SYDN00AUS, V2 sydn)
	\item[-{}-rinexversion \textless 2,3 \textgreater ]  rinex version of station observation files
	\item[-{}-system \textless SYSTEM \textgreater]      gnss system (GLONASS,BEIDOU,GPS,GALILEO,MIXED
	\item[-{}-noproxy]             disable use of proxy server
	\item[-{}-proxy \textless PROXY \textgreater]    set your proxy server (server:port)
	\item[-{}-version, -v ]       print version information and exit
\end{description*}

\subsubsection{examples}

This downloads V3 mixed observation files from the IGS station CEDU, with the identifier CEDU00AUS
for days 10 to 12 in 2020.\\
\begin{lstlisting}
fetchigs.py --centre CDDIS --observations --statid CEDU00AUS --version 3 --system MIXED 2010-10 2020-12
\end{lstlisting}

This downloads final IGS clock and orbit products for MJD 58606 from the CDDIS data centre.\\
\begin{lstlisting}
fetchigs.py --centre CDDIS --clocks --orbits --final 58606
\end{lstlisting}

This downloads a \cc{brdc} broadcast ephemeris file.\\
\begin{lstlisting}
fetchigs.py --centre GSSC --ephemeris --system MIXED --rinexversion 2 58606
\end{lstlisting}

\subsubsection{configuration file}

The \cc{[Main]} section has two keys.

{\bfseries Data centres}\\
This is a comma-separated list of sections which define IGS data centres which can be used to download data.\\
\textit{Example:}
\begin{lstlisting}
Data centres = CDDIS,GSSC
\end{lstlisting}

{\bfseries Proxy server}\\
This sets a proxy server (and port) to be used for downloads.\\
\textit{Example:}
\begin{lstlisting}
Proxy server = someproxy.in.megacorp.com:8080
\end{lstlisting}

Each defined IGS data centre has the following keys, defining various paths.

{\bfseries Base URL}\\
This sets the base URL for downloading files.\\
\textit{Example:}
\begin{lstlisting}
Base URL = ftp://cddis.gsfc.nasa.gov
\end{lstlisting}

{\bfseries Broadcast ephemeris}\\
This sets the path relative to the base URL for downloading broadcast ephemeris files.\\
\textit{Example:}
\begin{lstlisting}
Broadcast ephemeris = gnss/data/daily
\end{lstlisting}

{\bfseries Products}\\
This sets the path relative to the base URL for downloading IGS products.\\
\textit{Example:}
\begin{lstlisting}
Products = gnss/products
\end{lstlisting}

{\bfseries Station data}\\
This sets the path relative to the base URL for downloading IGS station RINEX files. \\
\textit{Example:}
\begin{lstlisting}
Station Data = gnss/data/daily
\end{lstlisting}


\subsection{ticqc.py}

\hypertarget{h:ticqc}{}

\cc{ticqc} checks TIC files. It currently reports the total number of measurements, duplicates and gaps in the file,
and the data range.

\subsubsection{usage}

\begin{lstlisting}[mathescape=true]
ticqc.py [option] file
\end{lstlisting}
The command line options are:
\begin{description*}
\item[-{}-help, -h] print help and exit
\item[-{}-verbose ] verbose output eg duplicate measurements are printed out
\item[-{}-version, -v] print version information and exit
\end{description*}


