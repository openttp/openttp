
This chapter describes software related to GPSCV time-transfer.


\section{Software overview}

Time-transfer files are produced by \cc{mktimetx} from the GNSS receiver logs and counter/timer measurements.
The time-transfer files are in  RINEX observation  and CGGTTS format. 
Currently, CGGTTS-format files are only produced for GPS. 
Figure \ref{f:GPSCVProcessing} illustrates the processing chain TODO. 

The OpenTTP software suite is catalogued in Table \ref{t:OTTPSoftware}
\begin{table}
\begin{tabular}{l|l|l}
	& program & \\ 
	\hline
GPSCV processing  &  & \\
	& \hyperlink{h:mktimetx}{mktimetx} & core program\\
	& \hyperlink{h:runmktimetx}{runmktimetx.pl} & \\
	& \hyperlink{h:mkcggtts}{mkcggts.py} & \\
	\hline
TIC logging & & \\
	& \hyperlink{h:hp5313xlog}{hp5313xlog.pl} &\\
	& \hyperlink{h:cnt9xlog}{cnt9xlog.py} &\\
	& \hyperlink{h:okxemlog}{okxemlog.pl} & \\
	& \hyperlink{h:prs10log}{prs10log.pl} & \\
	& \hyperlink{h:ticclog}{ticclog.py} & \\
	\hline
GNSS receiver logging & & \\
	&	\hyperlink{h:jnslog}{jnslog.pl} & Javad\\
	& \hyperlink{h:nvslog}{nv08log.pl} & NVS\\
	& \hyperlink{h:restlog}{restlog.pl} & Trimble Resolution T\\
	& \hyperlink{h:ublox9log}{ublox9log.py} & ublox\\
	& \hyperlink{h:ubloxlog}{ubloxlog.pl} & ublox\\
GNSS receiver utilities & & \\
	& \hyperlink{h:jnsextract}{jnsextract.pl} & \\
	& \hyperlink{h:nv08extract}{nv08extract.pl} & \\
	& \hyperlink{h:nv08info}{nv08info.pl} & \\
	& \hyperlink{h:restextract}{restextract.pl} & \\
	& \hyperlink{h:restinfo}{restinfo.pl} & \\
	& \hyperlink{h:ubloxextract}{ubloxextract.py} & \\
	& \hyperlink{h:ubloxmkdev}{ubloxmkdev.py} & \\
	\hline
Analysis tools & & \\
	& \hyperlink{h:cggttsqc}{cggttsqc.py} & \\
	& \hyperlink{h:cmpcggtts}{cmpcggtts.py} & \\
	& \hyperlink{h:editcggtts}{editcggtts.py} & \\
	& \hyperlink{h:editrnxnav}{editrnxnav.py} & \\
	& \hyperlink{h:editrnxobs}{editrnxobs.py} & \\
	& \hyperlink{h:ticqc}{ticqc.py} & \\
Miscellaneous & & \\
  & \hyperlink{h:log1Wtemp}{log1Wtemp.pl} & \\ 
	\hline
\end{tabular}
\caption{GPSCV software overview \label{t:OTTPSoftware} }
\end{table}

\section{crontab \label{ss:crontab}}

Automatic logging, processing and archival of data is co-ordinated via the user \cc{cvgps}' \cc{crontab}.

A minimal \cc{crontab} for the user \cc{cvgps} looks like this:
\begin{lstlisting}
# Check that all logging is running every 5 minutes
*/5 * * * * /usr/local/bin/kickstart.pl # See ~/etc/kickstart.conf

# Run the processing of the data at 00:15
15 0 * * * nice $HOME/bin/runmktimetx.pl >/dev/null 2>&1 

# Give the processing some time to complete, then zip the files at 00:45
45 0 * * * nice $HOME/bin/gziplogs.pl >/dev/null 2>&1

# Check the status of the system once a day, just before the day rollover
56 23 * * * $HOME/bin/checkstatus.pl >$HOME/lockStatusCheck/status.dat
\end{lstlisting}
showing the three essential processes of logging, processing and archival of data.

The active crontab can be examined with the command \cc{crontab -l}. 
A default \cc{crontab} is saved in \cc{/home/cvgps/etc/crontab}. 

\section{Configuration file format \label{sConfigFileFormat}}

Configuration files use a common format and are plain text files, designed to be easily edited via a command-line
editor because in many applications, only shell access to the system will be available.

The file is usually divided into sections, with section names delimited by square brackets [ ]. Entries in each section
are of the form:
\begin{lstlisting}
key = value
\end{lstlisting}
For example,
\begin{lstlisting}
[Receiver]
manufacturer = Trimble
model = Resolution T
\end{lstlisting}
defines a section \cc{Receiver} and the two keys: \cc{manufacturer} and \cc{model}. 

The notation \cc{Section::Key} is used to fully specify keys. For example,
\cc{Receiver::model} and \cc{Receiver::manufacturer} specify the two keys above.

Keys and section names are not case-sensitive. In particular, the python and Perl libraries
which provide functions for reading the configuration files convert keys and section names to lower case.
The case of key values is preserved, since this may be signficant eg path names.

Leading and trailing whitespace is removed from keys, key values and section names.

Comments begin with a `\#'. 

Some keys define a list of sections. 
For example, the comma-separated list of values for \cc{CGGTTS::outputs} 
\begin{lstlisting}
[CGGTTS]
outputs = c1-code,p1-code,p2-code
\end{lstlisting}
defines three sections: \cc{c1-code}, \cc{p1-code}, and \cc{p2-code}.
This is a bodge which would be more elegantly addressed using an extensible format like XML, 
but it has proven to be sufficient for our needs.

\subsection{Paths} \label{s:ConfigFilePaths}

Paths to files  specified in a configuration file are constructed with the following 
precedence:
\begin{enumerate}
\item If the path begins with a leading slash, then it is interpreted as an absolute path
\item If a non-absolute path is specified, it is interpreted as being relative to 
	the users's home directory (or where the configuration file allows specification of a different root path eg in \hyperlink{h:rootpath}{gpscv.conf}, relative to that root path) 
\item Otherwise, the default path is used.
\end{enumerate}
Most software in OpenTTP follows these conventions.

\section{Data file formats \label{s:DataFileFormat}}

\subsection{GPS receiver}

This text file records messages from the GPS receiver. 
The native format of the messages can be a mix of ASCII and binary messages.
Binary messages are hexadecimal-encoded for saving in the log file. 
Some logging scripts record ancillary information, such as commands sent to the receiver. 
Comments in the log file are allowed, prefaced by a `\#' character. 
The `@' character is used to tag lines containing special information that needs to be parsed by
the processing software.

Messages are successive lines of the form:
\begin{lstlisting}
<message_id> <time_stamp> <message>
\end{lstlisting}
\textit{Example:}
\begin{lstlisting}
TO 00:00:02 cdfbc75a9a8c353fc5
\end{lstlisting}

Hex encoding of binary messages results in much larger files but these compress to a size not much larger
than the original binary data.

The exception to this is the Septentrio logging script, which uses the native SBF format.
Some logging scripts have the option of logging data in the receiver's native format, to facilitate
use of other tools.

\subsection{Time-interval counter \label{s:TICformat}}

This text file records the difference between GNSS receiver and the Reference Oscillator 1 pps,
measured each second. The convention is that the Reference oscillator provides the start trigger.
Entries are successive lines of the form:
\begin{lstlisting}
<time_of_day> <time_difference>
\end{lstlisting}
where the time difference is in seconds.\\
\textit{Example:}
\begin{lstlisting}
00:00:04  +4.0821E-006 
\end{lstlisting}

\input{srcs/gpscvconf}

\input{srcs/mktimetx}

\input{srcs/runmktimetx}

\section{mkcggtts.py \label{s:mkcggtts}}

\hypertarget{h:mkcggtts}{}

\cc{mkcggtts.py} is used for scripting the generation of CGGTTS files from RINEX navigation and observation files using
a third party tool. Currently, the only tool supported is \cc{r2cggtts}. It will default to \cc{rnx2cggtts}, when this matures.

\subsection{usage}

\begin{lstlisting}[mathescape=true]
mkcggtts.py [OPTION] $\ldots$ [mjd [mjd $\ldots$]]
\end{lstlisting}
The command line options are:
\begin{description*}
	
	\item[-{}-help,-h]	print the help information and exit
	\item[-{}--config \textless{file}\textgreater, -c \textless{file}\textgreater] use the specified configuration file 
	\item[-{}-debug,-d]	print debugging information to stderr
	\item[-{}-leapsecs \textless{n}\textgreater]	set the number of leap seconds
	\item[-{}-previousmjd]	when no MJD (or one MJD) is given, MJD-1 is added to the MJDs to be processed
	\item[-{}-version,-v] show the version information and exit
\end{description*}

The leap second count in \cc{paramCGGTTS} is determined from the RINEX navigation file if possible.
Otherwise, it can be specified manually via  the \cc{--leapsecs} option.

The \cc{--previousMJD} option is intended to be used in automated daily processing using \cc{r2cggtts}, 
to reprocess data for any missed track at the end of the day.

\subsection{configuration file}

See the sample!

\subsection{examples}

None yet!


\input{srcs/cnt9xlog}

\input{srcs/hp53131xlog}

\input{srcs/okxemlog}

\input{srcs/prs10log}

\input{srcs/ticclog}

\section{Javad/Topcon receivers}


Javad GPS Receiver Interface Language (GRIL) receivers are obsolete and further development of the software
described here has ceased.
The software has been tested with:
Topcon Euro-80
Topcon Euro-160

\subsection{jnslog.pl}
\hypertarget{h:jnslog}{}

\cc{jnslog.pl} is used to configure and log Javad 
 GRIL receivers.

It doesn't produce a log file.

\subsubsection{usage}

\begin{lstlisting}[mathescape=true]
jnslog.pl [option] $\ldots$ configurationFile
\end{lstlisting}

The command line options are:
\begin{description*}
 \item[-h] print help and exit
 \item[-d] run in debugging mode
 \item[-r] suppress the reset of the receiver on startup (NOTE! automatic reset)
 \item[-v] print version information and exit
\end{description*}

\subsubsection{configuration}

The file \cc{receiver.conf} lists the commands used to configure the receiver.
For example, to configure a single frequency receiver:
\begin{lstlisting}
		# GRIL commands
    set,out/elm/dev/ser/a,$Init{receiver:elevation mask}
    set,pos/iono,off  
    set,pos/tropo,off 
    set,dev/pps/a/time,gps
    set,dev/pps/a/offs/ns,$Init{receiver:pps offset}
    set,dev/pps/a/per/ms,1000
    set,dev/pps/a/out,on
    
    # Receiver messages to turn on
    RT                  # Receiver time
    TO                  # Receiver base time to receiver time offset
    ZA                  # PPS offset (sawtooth)
    YA                  # Extra time offset 
    SI                  # Satellite index
    EL                  # Satellite elevation
    AZ                  # Satellite azimuth
    SS                  # Satellite navigation status
    FC                  # C/A lock loop status bits
    F1 
    F2 
    RC                  # Full pseudorange C/A
    R1                  # Full pseudorange P/L1
    R2                  # Full pseudorange P/L2
    P1                  # For RINEX Obs
    P2                  # For RINEX Obs
    RD                  # For RINEX
    NP                  # Navigation Position text message
    UO:{3600,14400,0,2} # UTC parameters, when changes or every 4 hrs
    IO:{3600,14400,0,2} # Ionospheric parameters
    GE:{1,3600,0,2}     # GPS ephemeris data
\end{lstlisting}
The GRIL commands in the first part of the file are executed verbatim, with subsitution
of values from the configuration file, written as Perl hash table lookups. 
Messages to be enabled are then listed, one per line. The message rate can be specified 
in the GRIL syntax.

Configuration commands in \cc{gpscv.conf} specific to this receiver are:

JNS receivers have a number of specific configuration entries in \cc{gpscv.conf}:
\begin{itemize}
\item \hyperlink{h:configuration}{receiver::configuration}
\item \hyperlink{h:pps_synchronization}{receiver::pps synchronization}
\item \hyperlink{h:pps_synchronization_delay}{receiver::pps synchronization delay}
\item \hyperlink{h:paths_rinex_l1l2}{paths::rinex l1l2}
\end{itemize}

\subsection{jnsextract.pl}
\hypertarget{h:jnsextract}{}

\cc{jnsextract.pl} is used to decode and extract information from Javda receiver log files. 
It will decompress the file if necessary.
It doesn't use \cc{gpscv.conf}. 

\subsubsection{usage}

\begin{lstlisting}[mathescape=true]
jnsextract.pl [option] $\ldots$ file
\end{lstlisting}

The command line options are:
\begin{description*}

	\item[-b \textless{value}\textgreater]  start time (s or hh:mm, default 0)
	\item[-c] compress (eg skip repeat values) if possible
	\item[-e \textless{value}\textgreater]  stop time (s or hh:mm, default 24:00)
	\item[-g] select GPS only
	\item[-i \textless{n}\textgreater]  decimation interval (in seconds, default=1)
	\item[-k]  keep the uncompressed file, if created
	\item[-o \textless{mode}\textgreater] output mode:
		\begin{description*}
			\item[bp]  receiver sync to reference time (BP message)
			\item[cn]  carrier-to-noise vs elevation
			\item[do]  DO derivative of time offset vs time
			\item[si]  visibility vs time (SI message)
			\item[np]  visibility vs time (NP message)
			\item[vt]  visibility vs time (list time for each SV)
			\item[tv]  visibility vs time (list SV for each time)
			\item[tr]  satellite tracks (PRN, azimuth, elevation)
			\item[tn]  satellite tracks (PRN, azimuth, elevation, CN)
			\item[to]  TO time offset vs time
			\item[ya]  YA time offset vs time
			\item[za]  ZA time offset vs time
			\item[st]  ST solution time tag vs time
			\item[uo]  UO UTC parameters
		\end{description*}
	\item[-r]    select GLONASS only

\end{description*}

\subsection{runrinexobstc.pl}

\hypertarget{h:runrinexobstc}{}

\cc{runrinexobstc.pl} runs the now deprecated \cc{rinexobstc}, which produces a
RINEX observation file suitable for precise positioning when used with a 
suitable Javad receiver. \cc{mktimetx} can now be configured to produce suitable output.

\cc{runrinexobstc.pl} doesn't produce a log file.
	
\subsubsection{usage}
\cc{runrinexobstc.pl} is normally run as a \cc{cron} job.

To run \cc{runrinexobstc.pl} on the command line, use
\begin{lstlisting}[mathescape=true]
runrinexobstc.pl [option] $\ldots$ [Start MJD  [Stop MJD]]
\end{lstlisting}

\cc{Start MJD} and \cc{Stop MJD} specify the range of MJDs to process.
If a single MJD is specified, then data for that day is processed. If no
MJD is specified, the previous day's data is processed.

The options are:
\begin{description*}
	\item[-a \textless{file}\textgreater]  extend check for missed processing back \cc{n} days 
		(the default is~7)
	\item[-c \textless{file}\textgreater] use the specified configuration file
	\item[-d]	run in debugging mode
	\item[-h]	print help and exit
	\item[-x] run missed processing
	\item[-v]	print version information and exit
\end{description*}

\subsubsection{configuration}

\cc{runrinexobstc.pl} uses \cc{gpscv.conf}. 

It uses the following specific configuration file entries in \cc{gpscv.conf}; most
relate to writing the RINEX observation file header:
\begin{itemize}
	\item \hyperlink{h:antenna_x}{antenna::x}
	\item \hyperlink{h:antenna_y}{antenna::y}
	\item \hyperlink{h:antenna_z}{antenna::z}
	\item \hyperlink{h:antenna_marker_name}{antenna::marker name}
	\item \hyperlink{h:antenna_marker_number}{antenna::marker number}
	\item \hyperlink{h:antenna_delta_e}{antenna::delta e}
	\item \hyperlink{h:antenna_delta_h}{antenna::delta h}
	\item \hyperlink{h:antenna_delta_n}{antenna::delta n}
	\item \hyperlink{h:antenna_antenna_number}{antenna::antenna number}
	\item \hyperlink{h:antenna_antenna_type}{antenna::antenna type}
	\item \hyperlink{h:rinex_agency}{rinex::agency}
	\item \hyperlink{h:rinex_observer}{rinex::observer}
	\item \hyperlink{h:paths_rinex_l1l2}{paths::rinex l1l2}
\end{itemize}


\section{NVS NV08C receivers}

This receiver has become difficult to buy in small quantities and development 
of the software described here has ceased.
\marginpar{\epsfig{file=figures/warning.eps,silent=,width=48pt}}

\subsection{nv08log.pl} \hypertarget{h:nvslog}{}

\cc{nv08log.pl} is used to configure and log NVS NV08 receivers.
It needs the Perl library \cc{NV08C}.

It doesn't produce a log file.

There are no NV08-specific configuration commands in \cc{gpscv.conf}.



\subsubsection{usage}

\begin{lstlisting}[mathescape=true]
nv08log.pl [option] $\ldots$ 
\end{lstlisting}

The command line options are:
\begin{description*}
	\item[-c \textless{file}\textgreater] use the specified configuration file
	\item[-h] print help and exit
	\item[-d] run in debugging mode
	\item[-r] reset the receiver on startup
	\item[-v] print version information and exit
\end{description*}

\subsection{nv08extract.pl} \hypertarget{h:nv08extract}{}

\cc{nv08extract.pl} is used to decode and extract information from NVS NV08 receiver log files. 
It will decompress the file if necessary.

It uses \cc{gpscv.conf} to construct receiver log file names if the file is not explicitly given.

\subsubsection{usage}

\begin{lstlisting}[mathescape=true]
nv08extract.pl [option] $\ldots$ [file]
\end{lstlisting}

Given an MJD via a command line option, it will construct a file name using the information in \cc{gpscv.conf}. If no MJD or file is given, it assumes the current MJD for the file. 

The command line options are:
\begin{description*}
	\item[-c \textless{file}\textgreater] use the specified configuration file
	\item[-h] print help and exit
	\item[-d] run in debugging mode
	\item[-v] print version information and exit
	\item[-m \textless{MJD}\textgreater]  MJD of the file to process
	\item[-t] extract Time, Date and Time Zone offset
	\item[-o] extract Receiver Operating Parameters
	\item[-s] extract Visible Satellites
	\item[-n] extract Number of Satellites and Dilution Of Precision (DOP)
	\item[-w] extract Software Version, Device ID and Number of Channels
	\item[-f] extract 1 PPS 'sawtooth' correction
	\item[-T] extract Time Synchronisation Operating Mode (antenna cable delay, averaging time)
	\item[-p] extract PVT Vectors and associated quality factors (including TDOP)
	\item[-a] extract Additional Operating Parameters
	\item[-P] extract Port status messages
	\item[-e] extract Satellite ephemeris
	\item[-l] extract Time scale parameters
	\item[-i] extract Ionosphere parameters
	\item[-g] extract GPS, GLONASS and UTC time scale parameters
	\item[-r] extract Raw data (pseudoranges, etc)
	\item[-G] extract Geocentric antenna coordinates in WGS-84 system
	\item[-u] extract Unknown message (garbage data)
	\item[-z] less verbose output
\end{description*}

\subsection{nv08info.pl} \hypertarget{h:nv08info}{}

This actually does nothing useful. If it did, it would query the receiver for its
serial number and so on. 

\subsubsection{usage}

\begin{lstlisting}[mathescape=true]
nv08info.pl [option] $\ldots$ 
\end{lstlisting}

The command line options are:
\begin{description*}
	\item[-c \textless{file}\textgreater] use the specified configuration file
	\item[-h] print help and exit
	\item[-d] run in debugging mode
	\item[-v] print version information and exit
\end{description*}


\section{Septentrio receivers}

A typical processing chain for the Septentrio receiver looks like:

\begin{enumerate}
\item \cc{plrxlog.py}    to log the receiver
\item \cc{runsbf2rnx.py} to generate RINEX files from SBF
\item \cc{mkcggtts.py}   to generate CGGTTS files from RINEX
\end{enumerate}

Currently, the processing chain relies on the tools provided by Septentrio and r2cggtts.
Open source replacements for these are currently under development.
OpenTTP now provides \cc{sbf2rnx} as a replacement for sbf2rin, but this is limited to GPS at present.

\subsection{plrxlog.py}
\hypertarget{h:plrxlog}{}

\cc{plrxlog.py} is used to configure and log Septentrio receivers.
Unlike the other receiver logging scripts, it logs using the receiver's native binary data format, rather
than the custom OpenTTP format.
It has mainly been used with mosaicT receivers.

It doesn't produce a log file.

\subsubsection{usage}

\begin{lstlisting}[mathescape=true]
plrxlog.py [option] $\ldots$ configurationFile
\end{lstlisting}

The command line options are:
\begin{description*}
	\item[--config \textless{file}\textgreater, -c \textless{file}\textgreater] use the specified configuration file
	\item[--debug, -d]	run in debugging mode
	\item[--reset, -r]  reset receiver and exit
	\item[--help, -h]	print help and exit
	\item[--version, -v]	print version information and exit
\end{description*}

A reset command issues:
\begin{lstlisting}
exeResetReceiver,Hard,PVTData+SatData
\end{lstlisting}

\subsubsection{configuration}

The file \cc{receiver.conf} lists any custom commands used to configure the receiver.
These are written in Septentrio XXX format, and passed verbatim to the receiver.
For example,
\begin{lstlisting}
lstInternalFile,Identification
SetPPSparameters,sec1,Low2High,0,RxClock,60
SetTimeSyncSource,EventA
SetSignalTracking,+GPSL1PY+BDSB2A+GALE5
\end{lstlisting}

The default configuration for the mosaicT enables SBF output and the message set needed for RINEX generation and the SatVisibility message.

\subsection{mosaicmkdev.py}

\cc{mosaicmkdev.py} is used to create a symbolic link for the mosaic-T receiver.
Currently, it identifies a receiver by the serial number of the USB hub
that the various available USB devices in the mosaic-T (development kit) are accessed through.
It is typically used with a \cc{udev} rule, an example of which is included in the OpenTTP distribution.

The default configuration file is \cc{/usr/local/etc/mosaicmkdev.conf} which looks like:
\begin{lstlisting}
[main]
receivers = mos01

[mos01]
serial number = 3612929

# Symlinks to be made
ttyACM0 = mos01ACM0
ttyACM1 = mos01ACM1
\end{lstlisting}

The \cc{receivers} option is a comma-separated list of section names, as per the usual convention.
Each section defines the setup for a particular receiver.

The serial number is currently matched through the \cc{udev} `root device' property `ID\_SERIAL\_SHORT'.

The symlinks to be made in \cc{/dev} are defined by the Linux device names (\cc{ttyACM0}, \cc{ttyACM1} only at present).

Startup errors reported via \cc{udev} can be examined using
\begin{lstlisting}
systemctl status systemd-udevd
\end{lstlisting}

\subsection{runsbf2rnx.py}

\cc{runsbf2rnx.py} is a wrapper for \cc{sbf2rin}, provided by Septentrio.

\subsubsection{usage}

\begin{lstlisting}[mathescape=true]
runsbf2rnx.py [OPTION] $\ldots$ [mjd [mjd $\ldots$]]
\end{lstlisting}
The command line options are:
\begin{description*}
	\item[-{}-help,-h]	print the help information and exit
	\item[-{}-config \textless{file}\textgreater, -c \textless{file}\textgreater] use the specified configuration file 
	\item[-{}-debug,-d]	print debugging information to stderr
	\item[-{}-version,-v] show the version information and exit
\end{description*}

\subsubsection{configuration}

{\bfseries [Main] section}\\

{\bfseries exec}\\
This specifies the path to the executable \cc{sbf2rin}.
The default is \cc{/usr/local/bin/sbf2rin}.

{\bfseries sbf station name}\\

{\bfseries [Receiver] section}\\

{\bfseries file extension}\\

{\bfseries [RINEX] section}\\

{\bfseries version}\\
This option specifies the RINEX version for the output, via the option to \cc{sbf2rin}

\textit{Example:}
\begin{lstlisting}
version = 3 
\end{lstlisting}

{\bfseries name format}\\

{\bfseries obs directory}\\

{\bfseries nav directory}\\

{\bfseries obs sta}\\

{\bfseries nav sta}\\

{\bfseries create nav file}\\

{\bfseries exclusions}\\
This option specifies GNSS to be excluded from the generated RINEX files via the `-x' option to \cc{sbf2rin}.
The GNSS are specified using the same syntax as \cc{sbf2rin} uses.\\
\textit{Example:}
\begin{lstlisting}
exclusions = ISJ 
\end{lstlisting}

{\bfseries fix header}\\
This option specifies whether or not the RINEX header should be edited.

{\bfseries header fixes}\\
This option specifies the file to use when fixing the RINEX header. The file specifies replacements for lines in the
RINEX header, and needs to be in RINEX format.

{\bfseries fix satellite count}\\
Some older versions of \cc{sbf2rin} incorrectly report the number of satellites in the RINEX observation file header
when GNSS are excluded via the `-x' option to \cc{sbf2rin}. The offending line in the RINEX header is removed, since
it is optional.\\
\textit{Example:}
\begin{lstlisting}
fix satellite count = yes 
\end{lstlisting}

{\bfseries [Paths] section}\\

{\bfseries receiver data}\\

{\bfseries tmp}\\

\subsection{sbf2rinbatch.py}

\cc{sbf2rinbatch.py} is used for batch processing of SBF files via Septentrio's tool \cc{sbf2rin}.

\subsection{mksephourly.py}

\cc{mksephourly.conf} is used for hourly generation of REFSYS from CGGTTS files. 
It is intended for `live' monitoring of time-transfer.
The script calls \cc{runsbf2rnx.py} to generate RINEX, and then \cc{mkcggtts.py} to generate CGGTTS.
A new file named MJD.dat is created each day, and updated by rewriting it each time \cc{mksephourly.py} is called.

The format of this file is:
\begin{lstlisting}
MJD TOD (in seconds) REFSYS (mean) number of tracks
\end{lstlisting}

The command line options are:
\begin{description*}
	\item[--config \textless{file}\textgreater, -c \textless{file}\textgreater] use the specified configuration file
	\item[--debug, -d]	run in debugging mode
	\item[--help, -h]	print help and exit
	\item[--version, -v]	print version information and exit
\end{description*}

The default configuration file is \cc{~/etc/mksephourly.conf} which looks like:
\begin{lstlisting}
[Main]
runsbf2rnx conf = etc/hourly.runsbf2rnx.conf
cggtts source = CGGTTS-GPS-P3
mkcggtts conf = etc/hourly.mkcggtts.conf
summary path = hourly_tt/summary
\end{lstlisting}

{\bfseries runsbf2rnx conf}\\
This defines the configuration file to be used by \cc{runsbf2rnx.py}.

{\bfseries mkcggtts conf}\\
This defines the configuration file to be used by \cc{mkcggtts.py}.

{\bfseries cggtts source}\\
This defines the section in the configuration file used by \cc{mkcggtts} for CGGTTS file generation.
It is used to locate and identify the CGGTTS files that REFSYS is extracted from.

{\bfseries summary path }\\
This defines where summary files are written to.

\subsection{sbf2rnx}
Experimental!


\section{Trimble Resolution T receivers}

Trimble Resolution T receivers are obsolete and further development of the software
described here has ceased. The successor to the Resolution T, the SMT 360, 
cannot currently be used for time-transfer because of changes in the message set. 
\marginpar{\epsfig{file=figures/warning.eps,silent=,width=48pt}}
It is possible though that a future TSIP-based receiver may be suitable for time-transfer.

\subsection{restlog.pl} \hypertarget{h:restlog}{}

\subsubsection{usage}

\begin{lstlisting}[mathescape=true]
restlog.pl [option] $\ldots$ 
\end{lstlisting}

The command line options are:
\begin{description*}
	\item[-c \textless{file}\textgreater] use the specified configuration file
	\item[-h] print help and exit
	\item[-d] run in debugging mode
	\item[-r] reset the receiver on startup
	\item[-v] print version information and exit
\end{description*}

\subsubsection{configuration}

\cc{restlog.pl} respects the \cc{receiver::model} configuration option.
The valid values are:
\begin{enumerate}
	\item Resolution T
	\item Resolution SMT 360 
\end{enumerate}

\subsection{restextract.pl} \hypertarget{h:restextract}{}

\cc{restextract.pl} is used to decode and extract information from Trimble receiver log files. 
It will decompress the file if necessary.

\subsubsection{usage}

\begin{lstlisting}[mathescape=true]
restextract.pl [option] $\ldots$ 
\end{lstlisting}

Given an MJD via a command line option, it will construct a file name using the information in \cc{gpscv.conf}. If no MJD is given, it assumes the current MJD for the file.

The command line options are:
\begin{description*}
	\item[-c \textless{file}\textgreater] use the specified configuration file
	\item[-h] print help and exit
	\item[-v] print version information and exit
  \item[-a] extract S/N for visible satellites 
	\item[-f] show firmware version
  \item[-l] leap second warning
  \item[-L] leap second info )
  \item[-m \textless{mjd}\textgreater]  MJD of the file to process
  \item[-p] position fix message
  \item[-r \textless{svn}\textgreater] pseudoranges for satellite with given SVN 
			(svn=999 reports all satellites)
  \item[-s] number of visible satellites
  \item[-t] temperature 
  \item[-u] UTC offset 
  \item[-x] alarms and gps decoding status
\end{description*}

\subsection{restinfo.pl} \hypertarget{h:restinfo}{}

\cc{restinfo.pl} communicates with the receiver configured in \cc{gpscv.conf},
polling it for information such as hardware, software and firmware versions.
The serial port communication speed must be 115200 baud. 

\subsubsection{usage}

\begin{lstlisting}[mathescape=true]
restinfo.pl [option] $\ldots$ 
\end{lstlisting}

The command line options are:
\begin{description*}
	\item[-c \textless{file}\textgreater] use the specified configuration file
	\item[-h] print help and exit
	\item[-d] run in debugging mode
	\item[-v] print version information and exit
\end{description*}

\subsection{restconfig.pl} \hypertarget{h:restconfig}{}

\cc{restconfig.pl} is used to configure the receiver serial port for 115200 baud and no parity bit.
The latter is necessary for using the SMT 360 with \cc{ntpd}.
The new configuration is written to flash memory.

\subsubsection{usage}

\begin{lstlisting}[mathescape=true]
restconfig.pl [option] $\ldots$ 
\end{lstlisting}

The command line options are:
\begin{description*}
	\item[-c \textless{file}\textgreater] use the specified configuration file
	\item[-h] print help and exit
	\item[-d] run in debugging mode
	\item[-v] print version information and exit
\end{description*}

The serial port device name is read from \cc{gpscv.conf} or the specified file. 

\subsection{restplayer.pl} \hypertarget{h:restplayer}{}

This is used to replay raw data files through a serial port, simulating the 
operation of the GNSS receiver. This is useful for testing the operation of 
\cc{ntpd}, for example. This script will have to be modified for individual 
use because it currently hard codes paths and device names. 


\input{srcs/ublox}


\section{Miscellaneous tools}

\subsection{cggttsqc.py}

\hypertarget{h:cggttsqc}{}

\cc{cggttsqc.py} performs various checks on CGGTTS files.
Running it on a single CGGTTS file with no options will produce output like:

\begin{tabular}{lrrrrrr}
File         & Tracks  & Short & Min SV & Max SV   &  High DSG  &  Low ELV \\
57401.cctf   &   706   &  71   &   3    &  11      & 0     &  0   \\
\end{tabular}

where:\\
`Tracks' is the total number of tracks\\
`Short'  is the number of tracks with length less than a specified threshold (default 780 s)\\
`Min SV' is the minimum number of SVs visible\\
`Max SV' is the maximum number of SVs visible \\
`High DSG'    is the number of tracks with DSG above a specified threshold (default 20 ns)\\
`Low ELV'    is the number of tracks with elevation below a specified threshold (default 10 degrees)\\

\subsubsection{usage}

\begin{lstlisting}[mathescape=true]
cggttsqc.py [OPTION] $\ldots$ file [file $ldots$]
\end{lstlisting}
The command line options are:
\begin{description*}
	\item[-{}-help,-h]	print the help information and exit
	\item[-{}-debug,-d]	print debugging information to stderr
	\item[-{}-nowarn]   suppress warnings (eg about missing files, bad formatting in CGGTTS files)
	\item[-{}-dsg \textless{value}\textgreater]  set the upper limit for acceptable DSG. The units are ns.
	\item[-{}-elevation \textless{value}\textgreater] set the lower limit for acceptable elevation. The units are degrees.
	\item[-{}-tracklength  \textless{value}\textgreater] set the lower limit for acceptable track length. The units are seconds.
	\item[-{}-checkheader] show when significant fields in the header change, for example the delays.
	\item[-{}-nosequence] do not interpret (two) input filenames as a sequence.
	\item[-{}-plotcount]  plot a histogram of satellite count at each scheduled track time
	\item[-{}-version,-v] show the version information and exit
\end{description*}

\subsubsection{examples}

Mutiple files can be checked and a sequence can be specified with two file names.
For example:
\begin{lstlisting}
cggttsqc.py --dsg 5 GZAA0157.834 GZAA0157.876
\end{lstlisting}
will report on all files between MJDs 57834 and 57876, indicating in the DSG field the number of tracks with DSG greater
than 5 ns. For the two file names to be interpreted as a sequence, they must:
\begin{enumerate}
	\item have names in v2E CGGTTS recommended format or in the format MJD.xxx
	\item be in the same directory
	\item have the same extension, if in the format MJD.xxx
\end{enumerate}


\subsection{cmpcggtts.py}

\hypertarget{h:cmpcggtts}{}

\cc{cmpcggtts.py} matches tracks in paired CGGTTS files. 
It can be used for time transfer and delay calibration. 
It reads V1, V2, V2E CGGTTS files. 
It also reads the raw 30 s files produced by r2cggtts.

\subsubsection{usage}

To run \cc{kickstart.py} on the command line, use:
\begin{lstlisting}[mathescape=true]
cmpcggtts.py [option] $\ldots$ refDir calDir firstMJD lastMJD
\end{lstlisting}
where\\

\begin{description*}
\item[refDir]    directory for reference receiver
\item[calDir]    directory for receiver being calibrated 
\item[firstMJD]  first MJD to be processed
\item[lastMJD]   last MJD  to be processed
\end{description*}

The command line options are:
\begin{description*}
	\item[-{}-help]            show this help message and exit
	\item[-{}-starttime \textless{time}\textgreater ] time of day in HH:MM:SS format
	to start processing (default 00:00:00)
	\item[-{}-stoptime \textless{time}\textgreater]  time of day in HH:MM:SS format
	to stop processing (default 23:59:00)
	\item[-{}--calfrc \textless{calfrc}\textgreater]       set the calibration FRC code (L1C,L3P,\ldots)
	\item[-{}--reffrc \textless{refrc}\textgreater]      set the reference FRC code (L1C,L3P,\ldots)
	\item[-{}-elevationmask \textless{value}\textgreater] elevation mask (in degrees, default 0.0)
	\item[-{}-mintracklength \textless{value}\textgreater] minimum track length (in s, default 750)
	\item[-{}-maxdsg \textless{value}\textgreater]       maximum DSG (in ns, default 20.0)
	\item[-{}-matchephemeris ]     match on ephemeris (default no)
	\item[-{}-cv]                  compare in common view (default)
	\item[-{}-aiv]                 compare in all-in-view
	\item[-{}-acceptdelays]        accept the delays (no prompts in delay calibration mode)
	\item[-{}-refintdelays \textless{REFINTDELAYS}\textgreater] search for given internal delays in reference eg ``GPS P1,GPS P2''
  \item[-{}-calintdelays \textless{CALINTDELAYS}\textgreater] search for given internal delays in cal eg ``GPS C2''
	\item[-{}-delaycal]            delay calibration mode
	\item[-{}-timetransfer]        time-transfer mode (default)
	\item[-{}-ionosphere]          use the ionosphere in delay calibration mode (default = not used)
	\item[-{}-useRefMSIO]          use the measured ionosphere (mdio is removed from refsys and msio is subtracted, useful for V1 CGGTTS)
  \item[-{}-useCalMSIO]          use the measured ionosphere (mdio is removed from refsys and msio is subtracted, useful for V1 CGGTTS)
	\item[-{}-checksum]           exit on bad checksum (default = warn only)
	\item[-{}-refprefix \textless{value}\textgreater] file prefix for reference receiver (default = MJD)
	\item[-{}-calprefix \textless{value}\textgreater] file prefix for calibration receiver (default = MJD)
	\item[-{}-refext \textless{value}\textgreater]       file extension for reference receiver (default = cctf)
	\item[-{}-calext \textless{value}\textgreater]       file extension for calibration receiver (default = cctf)
	\item[-{}-comment \textless{COMMENT}\textgreater]     set comment on displayed plot
	\item[-{}-debug, -d]           debug (to stderr)
	\item[-{}-nowarn]              suppress warnings
	\item[-{}-quiet]               suppress all output to the terminal
	\item[-{}-keepall]             keep tracks after the end of the day
	\item[-{}-version, -v]         show version and exit
\end{description*}

The default mode is time-transfer. In this mode, a linear fit to the time-transfer data is made and the 
fractional frequency error and REFSYS (evaluated at the midpoint of the data set) are outputted. 
The uncertainties are estimated from the linear fit.

In delay calibration mode, the presumption is that the data are for two receivers sharing a
common clock and operating on a short baseline. The ionosphere correction is removed by default
but can be retained via a command line option. Delays as reported in the CGGTTS header can be 
changed interactively; prompting for the new delays can be skipped with a command line option.

Data can be filtered by elevation, track length and DSG.

Matching on ephemeris (IODE) can also be enforced. 
This can reduce time-transfer noise when CGGTTS files produced by different programs are compared.

Three text files are produced.

\subsubsection{examples}

Basic common-view time transfer with CGGTTS files in the folders \cc{refrx} and \cc{remrx}:
\begin{lstlisting}[mathescape=true]
cmpcggtts.py refrx remrx 57555 57556
\end{lstlisting}

Delay calibration with filenames according to the CGGTTS V2E specification:
\begin{lstlisting}[mathescape=true]
cmpcggtts.py --delaycal --refprefix GZAU01 --calprefix GZAU99 refrx remrx 57555 57556
\end{lstlisting}









\input{srcs/editcggtts}

\input{srcs/editrnxnav}

\subsection{editrnxobs.py}

\hypertarget{h:editrnxobs}{}

\cc{editrnxobs.py} is used to edit V3 RINEX observation files. 
Currently, it can:
\begin{enumerate}
\item Remove observations for specified GNSS.
\item Concatenate a sequence of daily files into a single file. 
\item Fix up observations missing at the beginning of the day due to  data files being rotated at the beginning of the UTC day, and not the GPS day. Observations are moved from the preceding day
\end{enumerate}

Depending on the operation, some fixups in the header are made, including the observed GNSS, number of SV and
the times of first and the last observations.

Operations can be combined, for example concatenation with removal of specified GNSS.


\subsubsection{usage}

\begin{lstlisting}[mathescape=true]
editrnxobs.py [option] $\ldots$ file/MJD [file/MJD] 
\end{lstlisting}
The command line options are:
\begin{description*}
	\item[-{}-help, -h]            print the help information and exit
	\item[-{}-debug, -d]           print debugging information to stderr
	\item[-{}-catenate, -c ]       catenate input files
	\item[-{}-excludegnss, -x  \textless{BEGRIJ}\textgreater]   remove specified satellite system	
	\item[-{}-fixmissing ]         fix missing observations due to UTC/GPS day rollover mismatch
  
	\item[-{}-template \textless{dir}\textgreater]    template for RINEX file names
	\item[-{}-obsdir   \textless{dir}\textgreater]    RINEX file directory
	\item[-{}-tmpdir   \textless{dir}\textgreater]    directory for temporary files
  
	\item[-{}-output, -o \textless{path}\textgreater] write output to file or directory
	 
	\item[-{}-replace, -r]         replace edited file
 
	\item[-{}-backup, -b  ]        create backup of edited file
	\item[-{}-version, -v ]        show the version information and exit
\end{description*}

A sequence of daily files can be specified using the names of the first and  last files or by specifying a range of MJDs.
When using MJDs, a template for the filename must be given. This can be a V2 or V3 style name.
For V2 names use \cc{DDD} and \cc{YY} in the template for day of year and year.
For V3 names, use \cc{DDD} and \cc{YYYY}  in the template for day of year and year.

Input files can be compressed (\cc{gzip}, Unix \cc{compress} and Hatanaka). 

They will be recompressed after processing.
Do not include the compression extension in the template name.

\subsubsection{examples}

The following command
\begin{lstlisting}[mathescape=true]
editrnxobs.py --catenate --excludegnss IJ -o ALL.rnx  /home/gnss/RINEX/SYDN00AUS_R_20210400000_01D_30S_MO.gz /home/gnss/RINEX/SYDN00AUS_R_20210470000_01D_30S_MO.gz
\end{lstlisting}
concatenates the files from DOY 40 to 47 to the file \cc{ALL.rnx} and removes IRNSS and QZSS observations.



\input{srcs/fetchigs}

\input{srcs/ticqc}
