\section{gziplogs.py \label{s:gziplogs}}

\cc{gziplogs.py} is used to manage compression of log files. 
Typically, it will be run once per day, after UTC0. It requires 
a configuration file, \cc{gziplogs.conf}, which is expected to be
in the user's \cc{etc} directory and uses our standard format \ref{sConfigFileFormat}.

\cc{gziplogs.py} doesn't produce a log file.

\subsection{usage}

\cc{gziplogs.py} is normally run as a \cc{cron} job. To run it on the command line, use:
\begin{lstlisting}[mathescape=true]
gziplogs.py [option] $\ldots$ [MJD [MJD]]
\end{lstlisting}
The command line options are:
\begin{description*}
	\item[--config \textless{file}\textgreater, -c \textless{file}\textgreater] use the specified configuration file
	\item[--debug, -d]	run in debugging mode
	\item[--help, -h]	print help and exit
	\item[--version, -v]	print version information and exit
\end{description*}

Either a single MJD or range of MJDs can be given. 
If no MJD is given, the MJD of the previous day is used.

\subsection{configuration file}

A \cc{targets} entry is needed. Note that this is not defined within a section, for compatibility with older versions of this script.

{\bfseries targets}\\
This entry defines a comma-separated list of targets. 
Each target defines a section in the configuration file.\\
\textit{Example:}
\begin{lstlisting}
targets = ppslogs,ntpstats
\end{lstlisting}


Within each section defined by targets, the following entries are defined

{\bfseries files}\\
This entry defines a comma-separated list of files to compress. 
Three date specifications, delimited by parentheses, are recognized: YYYYMMDD, MJD and YYDOY.\\
\textit{Example:}
\begin{lstlisting}
files = raw/{MJD}.rx, raw/{MJD}.tic, raw/{YYYYMMDD}.dat
\end{lstlisting}

{\bfseries destination}\\
This optional entry defines a directory to move compressed files to\\
\textit{Example:}
\begin{lstlisting}
destination = archive
\end{lstlisting}

Note, for compatibility, the file format used by  \cc{gziplogs.pl}, the deprecated Perl version of \cc{gziplogs.py} is also supported.
