\section{Septentrio receivers}

A typical processing chain for the Septentrio receiver looks like:

\begin{enumerate}
\item \cc{plrxlog.py}    to log the receiver
\item \cc{runsbf2rnx.py} to generate RINEX files from SBF
\item \cc{mkcggtts.py}   to generate CGGTTS files from RINEX
\end{enumerate}

Currently, the processing chain relies on the tools provided by Septentrio and r2cggtts.
Open source replacements for these are currently under development.
OpenTTP now provides \cc{sbf2rnx} as a replacement for sbf2rin, but this is limited to GPS at present.

\subsection{plrxlog.py}
\hypertarget{h:plrxlog}{}

\cc{plrxlog.py} is used to configure and log Septentrio receivers.
Unlike the other receiver logging scripts, it logs using the receiver's native binary data format, rather
than the custom OpenTTP format.
It has mainly been used with mosaicT receivers.

It doesn't produce a log file.

\subsubsection{usage}

\begin{lstlisting}[mathescape=true]
plrxlog.py [option] $\ldots$ configurationFile
\end{lstlisting}

The command line options are:
\begin{description*}
	\item[--config \textless{file}\textgreater, -c \textless{file}\textgreater] use the specified configuration file
	\item[--debug, -d]	run in debugging mode
	\item[--reset, -r]  reset receiver and exit
	\item[--help, -h]	print help and exit
	\item[--version, -v]	print version information and exit
\end{description*}

A reset command issues:
\begin{lstlisting}
exeResetReceiver,Hard,PVTData+SatData
\end{lstlisting}

\subsubsection{configuration}

The file \cc{receiver.conf} lists any custom commands used to configure the receiver.
These are written in Septentrio XXX format, and passed verbatim to the receiver.
For example,
\begin{lstlisting}
lstInternalFile,Identification
SetPPSparameters,sec1,Low2High,0,RxClock,60
SetTimeSyncSource,EventA
SetSignalTracking,+GPSL1PY+BDSB2A+GALE5
\end{lstlisting}

The default configuration for the mosaicT enables SBF output and the message set needed for RINEX generation and the SatVisibility message.

\subsection{mosaicmkdev.py}

\cc{mosaicmkdev.py} is used to create a symbolic link for the mosaic-T receiver.
Currently, it identifies a receiver by the serial number of the USB hub
that the various available USB devices in the mosaic-T (development kit) are accessed through.
It is typically used with a \cc{udev} rule, an example of which is included in the OpenTTP distribution.

The default configuration file is \cc{/usr/local/etc/mosaicmkdev.conf} which looks like:
\begin{lstlisting}
[main]
receivers = mos01

[mos01]
serial number = 3612929

# Symlinks to be made
ttyACM0 = mos01ACM0
ttyACM1 = mos01ACM1
\end{lstlisting}

The \cc{receivers} option is a comma-separated list of section names, as per the usual convention.
Each section defines the setup for a particular receiver.

The serial number is currently matched through the \cc{udev} `root device' property `ID\_SERIAL\_SHORT'.

The symlinks to be made in \cc{/dev} are defined by the Linux device names (\cc{ttyACM0}, \cc{ttyACM1} only at present).

Startup errors reported via \cc{udev} can be examined using
\begin{lstlisting}
systemctl status systemd-udevd
\end{lstlisting}

\subsection{runsbf2rnx.py}

\cc{runsbf2rnx.py} is a wrapper for \cc{sbf2rin}, provided by Septentrio.

\subsubsection{usage}

\begin{lstlisting}[mathescape=true]
runsbf2rnx.py [OPTION] $\ldots$ [mjd [mjd $\ldots$]]
\end{lstlisting}
The command line options are:
\begin{description*}
	\item[-{}-help,-h]	print the help information and exit
	\item[-{}-config \textless{file}\textgreater, -c \textless{file}\textgreater] use the specified configuration file 
	\item[-{}-debug,-d]	print debugging information to stderr
	\item[-{}-version,-v] show the version information and exit
\end{description*}

\subsubsection{configuration}

{\bfseries [Main] section}\\

{\bfseries exec}\\
This specifies the path to the executable \cc{sbf2rin}.
The default is \cc{/usr/local/bin/sbf2rin}.

{\bfseries sbf station name}\\

{\bfseries [Receiver] section}\\

{\bfseries file extension}\\

{\bfseries [RINEX] section}\\

{\bfseries version}\\
This option specifies the RINEX version for the output, via the option to \cc{sbf2rin}

\textit{Example:}
\begin{lstlisting}
version = 3 
\end{lstlisting}

{\bfseries name format}\\

{\bfseries obs directory}\\

{\bfseries nav directory}\\

{\bfseries obs sta}\\

{\bfseries nav sta}\\

{\bfseries create nav file}\\

{\bfseries exclusions}\\
This option specifies GNSS to be excluded from the generated RINEX files via the `-x' option to \cc{sbf2rin}.
The GNSS are specified using the same syntax as \cc{sbf2rin} uses.\\
\textit{Example:}
\begin{lstlisting}
exclusions = ISJ 
\end{lstlisting}

{\bfseries fix header}\\
This option specifies whether or not the RINEX header should be edited.

{\bfseries header fixes}\\
This option specifies the file to use when fixing the RINEX header. The file specifies replacements for lines in the
RINEX header, and needs to be in RINEX format.

{\bfseries fix satellite count}\\
Some older versions of \cc{sbf2rin} incorrectly report the number of satellites in the RINEX observation file header
when GNSS are excluded via the `-x' option to \cc{sbf2rin}. The offending line in the RINEX header is removed, since
it is optional.\\
\textit{Example:}
\begin{lstlisting}
fix satellite count = yes 
\end{lstlisting}

{\bfseries [Paths] section}\\

{\bfseries receiver data}\\

{\bfseries tmp}\\

\subsection{sbf2rinbatch.py}

\cc{sbf2rinbatch.py} is used for batch processing of SBF files via Septentrio's tool \cc{sbf2rin}.

\subsection{mksephourly.py}

\cc{mksephourly.conf} is used for hourly generation of REFSYS from CGGTTS files. 
It is intended for `live' monitoring of time-transfer.
The script calls \cc{runsbf2rnx.py} to generate RINEX, and then \cc{mkcggtts.py} to generate CGGTTS.
A new file named MJD.dat is created each day, and updated by rewriting it each time \cc{mksephourly.py} is called.

The format of this file is:
\begin{lstlisting}
MJD TOD (in seconds) REFSYS (mean) number of tracks
\end{lstlisting}

The command line options are:
\begin{description*}
	\item[--config \textless{file}\textgreater, -c \textless{file}\textgreater] use the specified configuration file
	\item[--debug, -d]	run in debugging mode
	\item[--help, -h]	print help and exit
	\item[--version, -v]	print version information and exit
\end{description*}

The default configuration file is \cc{~/etc/mksephourly.conf} which looks like:
\begin{lstlisting}
[Main]
runsbf2rnx conf = etc/hourly.runsbf2rnx.conf
cggtts source = CGGTTS-GPS-P3
mkcggtts conf = etc/hourly.mkcggtts.conf
summary path = hourly_tt/summary
\end{lstlisting}

{\bfseries runsbf2rnx conf}\\
This defines the configuration file to be used by \cc{runsbf2rnx.py}.

{\bfseries mkcggtts conf}\\
This defines the configuration file to be used by \cc{mkcggtts.py}.

{\bfseries cggtts source}\\
This defines the section in the configuration file used by \cc{mkcggtts} for CGGTTS file generation.
It is used to locate and identify the CGGTTS files that REFSYS is extracted from.

{\bfseries summary path }\\
This defines where summary files are written to.

\subsection{sbf2rnx}
Experimental!
