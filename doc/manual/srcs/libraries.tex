
\section{libraries}

\subsection{libconfigurator}

This is a C library used for parsing of configuration files. 
It is mainly used by \cc{mktimetx}.

The configuration file is read using
\begin{lstlisting}
int configfile_parse_as_list(ListEntry **first,const char *filename)
\end{lstlisting}
This returns a pointer to the first item in a linked list of \cc{ListEntry}, which describe entries in the configuration file.

Functions are provided for searching the list for an entry by section and key (token) name, returning a value of the
required type.  

\begin{lstlisting}
int list_get_int(ListEntry *first,const char *section,const char * token,int *value);
int list_get_float(ListEntry *first,const char *section,const char * token,float *value);
int list_get_double(ListEntry *first,const char *section,const char * token,double *value);
int list_get_string(ListEntry *first,const char *section,const char * token,char **value);
\end{lstlisting}

The return value of the function flags whether or not an error occurred.
The last error to occur is determined by calling:
\begin{lstlisting}
config_file_get_last_error()
\end{lstlisting}

The error codes are:
\begin{lstlisting}
NoError
FileNotFound
SectionNotFound
InternalError
ParseFailed
TokenNotFound
\end{lstlisting}

\cc{list\_clear()} can be used to delete the list.

This should be replaced one day by something based on the C++ STL!

\subsection{TFLibrary.pm}

This library is effectively deprecated because all new development is now in Python.

\subsection{OpenOK2}

This is a C++ library that can be used for communicating with the Opal Kelly FPGA board. It is 
a fork of a library originally written by Jennifer Holt and now manintained by Jorge Francisco.
Opal Kelly provides a similar library for a number of platforms. When OpenTTP development began,
ARM support was not available hence the use of OpenOK2. One significant limitation 
of the library is that it does not provide support for USB block transfers.

\subsection{ottplib.py}

This library replaces \cc{TFLibrary.pm} and should be used for all new development.

ottplib.{\bfseries{LoadConfig}}(\textit{path, options=\{\}})

Loads the configuration file specified by \cc{path}, returning key/value pairs as
a dictionary. Two options can be supplied:
\begin{description*}
\item[tolower] all keys are converted to lower case if this is \cc{true}
\item[defaults] a default configuration file is loaded first
\end{description*}
Example:
\begin{lstlisting}
cfg=ottplib.LoadConfig(configFile,{'tolower':True})
\end{lstlisting}

ottplib.{\bfseries{MJD}}(\textit{time})

Returns the Modified Julian Date, given the time in seconds since the epoch. It is usually called 
with \cc{time.time()}

ottplib.{\bfseries{MakeAbsolutePath}}(\textit{path,root})\\
Returns a path constructed relative to \cc{root}, unless \cc{path} is already absolute.
A trailing slash is added if it is absent.

ottplib.{\bfseries{MakeAbsoluteFilePath}}(\textit{file,home,defaultPath})

Returns an absolute path to \cc{file} according to the following  rules
\begin{enumerate}
	\item If \cc{path} begins with a slash then it is returned unchanged.
	\item If \cc{path} ends with a slash, then \cc{home} is prepended.
	\item Otherwise, \cc{defaultPath} is prepended.
\end{enumerate}

ottplib.{\bfseries{CreateProcessLock}}(\textit{file})

Creates a lock file named \cc{file}. 
The contents of the file are the process name and its PID.
When attempting to create a lock, the PID is checked in \cc{proc}.
Returns \cc{True} if the lock was successfully created.

ottplib.{\bfseries{RemoveProcessLock}}(\textit{file})\\
Removes the lock file named \cc{file}.
It returns nothing.

\subsection{cggttslib.py}

This library provides some functions for reading and manipulating CGGTTS files.

cggttslib.{\bfseries{CheckSum}}(\textit{string})

This computes the checksum of \cc{string} using the algorithm described in
the CGGTTS specification. The checksum is returned as an integer.

cggttslib.{\bfseries{ReadHeader}}(\textit{file})

This reads the CGGTTS header of \cc{file} and returns the fields in a Python dictionary.
Dictionary keys are the same as in the header. For example, \cc{REV DATE} 
has the key 'rev date' associated with it. The exceptions to this are the delays
\cc{INT DLY}, \cc{SYS DLY} and \cc{TOT DLY}. For convenience, these are further
broken down.

The function takes one argument, a file name and returns a list containing
the dictionary and any error messages as a single string. 
An empty dictionary is returned if:
\begin{enumerate}
\item the file cannot be opened
\item there is a formatting error in the header
\end{enumerate}
If the calculated checksum does not match the value in the header, an error message is returned.

cggttslib.{\bfseries{MakeFileSequence}}(\textit{file1,file2})

\cc{MakeFileSequence} interpolates a sequence of file names between \cc{file1} and \cc{file2}.
For example, if the two input file names are \cc{GZAU0158.532} and \cc{GZAU0158.537},
file names with MJDs ranging from 58532 through to 58537 are generated. The generated file names are 
returned as a Python list.

File names in two formats can be interpolated:
\begin{enumerate}
\item names in the format \cc{MJD.ext} where \cc{MJD} is a five digit MJD and \cc{ext}, the file extension, is arbitrary.
\item the format recommended in the CGGTTS v2E specification eg GZAU0158.532 
\end{enumerate}

For a sequence to be generated, the two file names must:
\begin{enumerate}
\item have the same path
\item have the same extension
\end{enumerate}
