\subsection{getgnssproducts.py}

\hypertarget{h:fetchigs}{}

\cc{getgnssproducts.py} uses the Python library \cc{request} to download GNSS products and station observation
files from IGS data centres via http. You will need a configuration file that sets up downloads from at least one IGS data
centre. The sample configuration file should be sufficient for most uses. 
Note that IGS data centres are increasingly requiring authenticated login so you may need to set this up.

\subsubsection{usage}

\begin{lstlisting}[mathescape=true]
getgnssproducts.py [option] $\ldots$ [start] [stop]
\end{lstlisting}

The \cc{start} and \cc{stop} times can be in the format:
\begin{description*}
\item[MJD] Modified Julian Date
\item[yyyy-doy] year and day of year (1 \textellipsis)
\item[yyyy-mm-dd] year, month (1-12) and day (1 \textellipsis)
\end{description*}

The command line options are:
\begin{description*}
	\item[-{}-help, -h] print help and exit
	\item[-{}-config \textless{file}\textgreater] , -c use this configuration file
	\item[-{}-debug, -d]           print debugging output to stderr
	\item[-{}-outputdir \textless{dir}\textgreater] set the output directory
	\item[-{}-ndays \textless{N}\textgreater] download previous N days
	
	\item[-{}-centre \textless{centre}\textgreater]       set data centres
	\item[-{}-listcentres, -l]     list available data centres
	\item[-{}-ephemeris]           get broadcast ephemeris
	\item[-{}-observations]        get station observations (only mixed observations for V3)
	\item[-{}-statid \textless{statid}\textgreater]      station identifier (eg V3 SYDN00AUS, V2 sydn)
	\item[-{}-rinexversion \textless{2,3}\textgreater ]  rinex version of station observation files
	\item[-{}-system \textless{system}\textgreater]      gnss system (GLONASS,BEIDOU,GPS,GALILEO,MIXED)
	
	\item[-{}-clocks]              get clock products (.clk)
	\item[-{}-orbits]              get orbit products (.sp3)
	\item[-{}-erp]                 get ERP products (.erp)
	\item[-{}-bias]                get bias products (BSX/DCB)
	\item[-{}-biasformat   \textless{format}\textgreater] format of downloaded bias products (BSX/DCB)
	\item[-{}-ppp]                 get products needed for PPP
	\item[-{}-rapiddir \textless{dir}\textgreater]   output directory for rapid products
  \item[-{}-finaldir \textless{dir}\textgreater]   output directory for final products
  \item[-{}-biasdir  \textless{dir}\textgreater]   output directory for bias products
  \item[-{}-force, -f ]          force download, overwriting existing files
	\item[-{}-rapid]               get rapid products
	\item[-{}-final]               get final products
	
	\item[-{}-noproxy]             disable use of proxy server
	\item[-{}-proxy \textless{proxy}\textgreater]    set your proxy server (server:port)
	\item[-{}-version, -v ]       print version information and exit
\end{description*}

\subsubsection{examples}

This downloads V3 mixed observation files from the IGS station CEDU, with the identifier CEDU00AUS
for MJDs 60300  to 60306.\\
\begin{lstlisting}
getgnssproducts.py --centre CDDIS --observations --statid CEDU00AUS --version 3 --system MIXED 60300 60306
\end{lstlisting}

This downloads final IGS clock and orbit products for MJD 58606 from the CDDIS data centre.\\
\begin{lstlisting}
getgnssproducts.py --centre CDDIS --clocks --orbits --final 58606
\end{lstlisting}

This downloads a \cc{brdc} broadcast ephemeris file.\\
\begin{lstlisting}
getgnssproducts.py --centre CDDIS --ephemeris --system MIXED --rinexversion 2 58606
\end{lstlisting}

\subsubsection{configuration file}

The \cc{[Main]} section has two keys.

{\bfseries Data centres}\\
This is a comma-separated list of sections which define IGS data centres which can be used to download data.\\
\textit{Example:}
\begin{lstlisting}
Data centres = CDDIS,GSSC
\end{lstlisting}

{\bfseries Proxy server}\\
This sets a proxy server (and port) to be used for downloads.\\
\textit{Example:}
\begin{lstlisting}
Proxy server = someproxy.in.megacorp.com:8080
\end{lstlisting}

Each defined IGS data centre has the following keys, defining various paths.

{\bfseries Base URL}\\
This sets the base URL for downloading files.\\
\textit{Example:}
\begin{lstlisting}
Base URL = https://cddis.nasa.gov/archive/gnss
\end{lstlisting}

{\bfseries Broadcast ephemeris}\\
This sets the path relative to the base URL for downloading broadcast ephemeris files.\\
\textit{Example:}
\begin{lstlisting}
Broadcast ephemeris =data/daily
\end{lstlisting}

{\bfseries Products}\\
This sets the path relative to the base URL for downloading IGS products.\\
\textit{Example:}
\begin{lstlisting}
Products = products
\end{lstlisting}

{\bfseries Station data}\\
This sets the path relative to the base URL for downloading IGS station RINEX files. \\
\textit{Example:}
\begin{lstlisting}
Station Data = data/daily
\end{lstlisting}

{\bfseries Bias}\\
This sets the path relative to the base URL for downloading IGS differential code bias files. \\
\textit{Example:}
\begin{lstlisting}
Bias  = products/bias
\end{lstlisting}
