\section{okcounterd}

\cc{okcounterd} provides the interface to the Opal Kelly FPGA development board when configured as a multichannel counter.
It communicates with user processes via port 21577.

\cc{okcounterd} recognizes the following commands, sent as plain text:
\begin{description*}
	\item[CONFIGURE]
	\item[QUERY CONFIGURATION]
	\item[LISTEN]
\end{description*}

\subsection{usage}
\cc{okcounterd} is automatically started by the system's init system. On Debian, this is \cc{systemd}. It can be run on
the command line for debugging purposes. The command line options are
\begin{description*}
	\item[-b] \<bitfile\> load the specified bitfile (the full path is needed)
	\item[-d]	run in debugging mode
	\item[-h]	print help and exit
	\item[-v]	print version information and exit
\end{description*}
To run \cc{okcounterd} on the command line, you will need to disable the system service
and kill any running \cc{okcounterd} process.

\subsection{configuration file \label{confformat}}
\cc{okcounterd} doesn't use a configuration file.

\subsection{log file}
\cc{okcounterd} doesn't produce a log file.