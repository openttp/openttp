\section{gpscv.conf - the core configuration file \label{sgpscvconf} }

A single configuration file, \cc{gpscv.conf}, provides configuration information to most of the
OpenTTP software. 
\cc{gpscv.conf} is used by \cc{mktimetx}, receiver logging scripts, TIC logging scripts,receiver utilities and so on.

It uses the format described in \ref{sConfigFileFormat}.

\begin{table}[ht]
\begin{tabular}{l|p{10cm}}
Section & Key \\ \hline
\hyperlink{h:antenna}{Antenna} & antenna number, antenna type, 
				delta H, delta N, delta E, frame,
				marker name, marker number,
				X, Y, Z\\ \hline
\hyperlink{h:cggtts}{CGGTTS} & BIPM cal id, comments, create, 
         ephemeris, ephemeris file, ephemeris path,
         internal delay, lab id, maximum DSG, minimum elevation,
         minimum track length, naming convention, outputs, reference,
         receiver id, revision date, version,
         code, constellation, path
         \\ \hline
\hyperlink{h:delays}{Delays}  & antenna cable, reference cable 
         \\ \hline
\hyperlink{h:counter}{Counter} & file extension, GPIB address, header generator, lock file,
         logger, logger options, okxem channel, port
				\\ \hline
\hyperlink{h:misc}{Misc}    & gzip
				\\ \hline
\hyperlink{h:paths}{Paths} & CGGTTS, counter data, processing log, receiver data, RINEX, tmp
				\\ \hline
\hyperlink{h:receiver}{Receiver} & configuration, elevation mask, logger, logger options, 
				 manufacturer, model, observations, 
         port, pps offset, synchronization, pps synchronization delay,
         status file, timeout, version
				\\ \hline
\hyperlink{h:reference}{Reference} & file extension, logging interval, log path, log status, oscillator, power flag, status file
        \\ \hline
\hyperlink{h:rinex}{RINEX}  & agency, create, observer, version
				\\ \hline
\end{tabular}
\caption{Summary of \cc{gpscv.conf} entries}
\end{table}

\subsection{[Antenna] section}

Entries used to create the RINEX header are:
\begin{itemize}
\item antenna number
\item antenna type
\item delta H, delta E, delta N
\item marker name
\item marker number
\item X,Y,Z
\end{itemize}

Entries used to create the CGGTTS header are:
\begin{itemize}
\item X,Y,Z
\end{itemize}

\hypertarget{h:antenna}{}
{\bfseries antenna number}\\
This appears as ANTNUM in the RINEX header.\\
\textit{Example:}
\begin{lstlisting}
antenna number = A567456
\end{lstlisting}

{\bfseries antenna type}\\
This appears as ANTTYPE in the RINEX header.\\
\textit{Example:}
\begin{lstlisting}
antenna type = Ashtec
\end{lstlisting}

{\bfseries delta H}\\
This appears as DELTA H in the RINEX header.\\
\textit{Example:}
\begin{lstlisting}
delta H = 0.0
\end{lstlisting}

{\bfseries delta E}\\
This appears as DELTA E in the RINEX header.\\
\textit{Example:}
\begin{lstlisting}
delta E = 0.0
\end{lstlisting}

{\bfseries delta N}\\
This appears as DELTA N in the RINEX header.\\
\textit{Example:}
\begin{lstlisting}
delta N = 0.0
\end{lstlisting}

{\bfseries frame}\\
This appears as FRAME in the CGGTTS header.\\
\textit{Example:}
\begin{lstlisting}
frame = ITRF2010
\end{lstlisting}

{\bfseries marker name}\\
This appears as MARKER NAME in the RINEX header.\\
\textit{Example:}
\begin{lstlisting}
marker name =
\end{lstlisting}

{\bfseries marker number}\\
This appears as MARKER NUMBER in the RINEX header.\\
\textit{Example:}
\begin{lstlisting}
marker number =
\end{lstlisting}

{\bfseries X}\\
This appears as X in the CGGTTS header and APPROX POSITION XYZ in the RINEXheader.\\
\textit{Example:}
\begin{lstlisting}
X = +4567890.123
\end{lstlisting}

{\bfseries Y}\\
This appears as Y in the CGGTTS header and APPROX POSITION XYZ in the RINEXheader.\\
\textit{Example:}
\begin{lstlisting}
Y = +2345678.90
\end{lstlisting}

{\bfseries Z}\\
This appears as Z in the CGGTTS header and APPROX POSITION XYZ in the RINEXheader.\\
\textit{Example:}
\begin{lstlisting}
Z = -1234567.890 
\end{lstlisting}


\subsection{[CGGTTS] section }

\hypertarget{h:cggtts}{Entries} in this section control the format and content of CGGTTS files 
and filtering applied to CGGTTS tracks in the final output.

To create CGGTTS output you must enable it: 


{\bfseries outputs}\\
This defines a list of sections which in turn define the desired CGGTTS outputs.\\
\textit{Example:}
\begin{lstlisting}
outputs = CGGTTS-GPS-C1,CGGTTS-GPS-P1,CGGTTS-GPS-P2
\end{lstlisting}

A CGGTTS v2E header looks like:
\begin{lstlisting}
CGGTTS     GENERIC DATA FORMAT VERSION = 2E
REV DATE = 2018-03-20
RCVR = NVS NV08 undefined 1999 mktimetx,v0.1.4
CH = 32
IMS = 99999
LAB = NMIA
X = -4648239.852 m
Y = +2560635.623 m
Z = -3526317.023 m
FRAME = ITRF2008
COMMENTS = none
INT DLY = 0.0 ns (GPS C1)     CAL_ID = none
CAB DLY = 0.0 ns
REF DLY = 0.0 ns
REF = UTC(AUS)
CKSUM = 1A
\end{lstlisting}

Entries in the header than can be defined in the CGGTTS section are as below. 

{\bfseries comments}\\
This defines a single COMMENTS line in the CGGTTS header.
The output line will be truncated at 128 characters.\\
\textit{Example:}
\begin{lstlisting}
comments = none
\end{lstlisting}

{\bfseries lab }\\
This defines the LAB line.\\
\textit{Example:}
\begin{lstlisting}
lab = NMI
\end{lstlisting}

{\bfseries reference}\\
This defines REF in the CGGTTS header.\\
\textit{Example:}
\begin{lstlisting}
reference = UTC(XXX)
\end{lstlisting}

{\bfseries revision date}\\
This defines REV DATE in the CGGTTS header. It must be in the format YYYY-MM-DD.\\
\textit{Example:}
\begin{lstlisting}
revision date = 2015-12-31
\end{lstlisting}


{\bfseries create}\\
This defines whether or not CGGTTS files will be generated by \cc{mktimetx}.\\
\textit{Example:}
\begin{lstlisting}
create = yes
\end{lstlisting}

{\bfseries ephemeris}\\
This defines whether to use the receiver-provided ephemeris or a user-provided ephemeris (via a RINEX navigation file).
If a user-provided ephemeris is specified then \cc{ephemeris path} and \cc{ephemeris file} 
must also be specified.\\
\textit{Example:}
\begin{lstlisting}
ephemeris = receiver
\end{lstlisting}

{\bfseries ephemeris file}\\
This specifies a pattern for user-provided RINEX navigation files.
Currently, only patterns of the form \cc{XXXXddd0.yyn} are recognized.\\
\textit{Example:}
\begin{lstlisting}
ephemeris file = brdcddd0.yyn
\end{lstlisting}

{\bfseries ephemeris path}\\
This specifies the path to user-provided RINEX navigation files.\\
\textit{Example:}
\begin{lstlisting}
ephemeris path = igsproducts
\end{lstlisting}

{\bfseries lab id}\\
This defines the two-character lab code used for creating BIPM-style file names, as per the V2E specification.\\
\textit{Example:}
\begin{lstlisting}
lab id = AU
\end{lstlisting}

{\bfseries maximum DSG}\\
CGGTTS tracks with DSG lower than this will be filtered out. 
The units are ns.\\
\textit{Example:}
\begin{lstlisting}
maximum DSG = 10.0
\end{lstlisting}

{\bfseries minimum elevation}\\
CGGTTS tracks with elevations lower than this will be filtered out. 
The units are degrees.\\
\textit{Example:}
\begin{lstlisting}
minimum elevation = 10
\end{lstlisting}

{\bfseries minimum track length}\\
CGGTTS tracks shorter than this will be filtered out. Tracks meeting the criterion are not necessarily contiguous.
The units are seconds.\\
\textit{Example:}
\begin{lstlisting}
minimum track length = 390
\end{lstlisting}

{\bfseries maximum URA}\\
GPS ephemerides with URA greater than this will not be used. 
They will still be written to RINEX navigation files, however.\\
\textit{Example:}
\begin{lstlisting}
maximum URA = 3.0
\end{lstlisting}

{\bfseries naming convention}\\
Defines the CGGTTS file naming convention. Valid options are `plain' (MJD.cctf) and `BIPM' styles.
The \cc{lab id} and \cc{receiver id} should be defined in conjunction with BIPM-style filenames.\\
\textit{Example:}
\begin{lstlisting}
naming convention = BIPM
\end{lstlisting}

{\bfseries receiver id}\\
This defines the two-character receiver code used for creating BIPM-style file names, 
as per the V2E specification.\\
\textit{Example:}
\begin{lstlisting}
receiver is = 01
\end{lstlisting}

{\bfseries version}\\
This defines the version of CGGTTS output. Valid versions are v1 and v2E. 
The \cc{lab id} and \cc{receiver id} should be defined in conjunction with v2E ouput\\
\textit{Example:}
\begin{lstlisting}
version = v2E
\end{lstlisting}


\subsubsection{CGGTTS output sections}

Multiple CGGTTS outputs can be defined, allowing for different constellation and signal combinations.
An example of a CGGTTS output section is as follows:
\begin{lstlisting}
[CGGTTS-GPS-C1]
constellation=GPS
code=C1
path=cggtts
BIPM cal id = none
internal delay = 11.0
\end{lstlisting}

{\bfseries BIPM cal id}\\
This defines CAL\_ID for the internal delay, as used in v2E CGGTTS headers.\\
\textit{Example:}
\begin{lstlisting}
BIPM cal id = none
\end{lstlisting}

{\bfseries code}\\

CGGTTS codes can be specified using either the two letter code, for example `C1' as described in the CGGTTS V2E 
specification, or three letter RINEX v3.03 observation codes.
\begin{table}
\begin{tabular}{ll}
	CGGTTS name	 & RINEX name \\ \hline
	C1 & C1C \\
	P1 & C1P \\
	E1 & C1C \\
	B1 & C2I \\
	C2 & C2C \\
	P2 & C2P \\
	B2 & C7I 
\end{tabular}
\caption{Correspondence of CGGTTS and RINEX signal names.}
\end{table}
\textit{Example:}
\begin{lstlisting}
code = P1+P3
\end{lstlisting}

{\bfseries constellation}\\
This defines the GNSS constellation. Only GPS is supported currently for CGGTTS generation.\\
\textit{Example:}
\begin{lstlisting}
constellation = GPS
\end{lstlisting}

{\bfseries internal delay}\\
This defines INT DLY in the CGGTTS header. The units are ns.\\
\textit{Example:}
\begin{lstlisting}
INT DLY = 0.0
\end{lstlisting}

{\bfseries path}\\
This defines the directory in which output files are placed.\\
\textit{Example:}
\begin{lstlisting}
path = cggtts
\end{lstlisting}

In v1 CGGTTS files, the delays are specified via INT DLY, CAB DLY and ANT DLY.  
For v2E files, the delays may be specified via the `system delay' and `total delay'.
If multiple delays (eg both internal and system delay are present) are defined in
\cc{gpscv.conf}, the precedence order is internal, system and then total delay.

The entries used to specify system and total delay are:\\

{\bfseries system delay}\\
This defines SYS DLY in the CGGTTS header. The units are ns.\\
\textit{Example:}
\begin{lstlisting}
SYS DLY = 0.0
\end{lstlisting}

{\bfseries total delay}\\
This defines TOT DLY in the CGGTTS header. The units are ns.\\
\textit{Example:}
\begin{lstlisting}
TOT DLY = 0.0
\end{lstlisting}

\subsection{[Counter] section}

\hypertarget{h:counter}{}

{\bfseries file extension}\\
This defines the extension used for time interval measurement files.
The default is `tic'.\\
\textit{Example:}
\begin{lstlisting}
file extension = tic
\end{lstlisting}

{\bfseries GPIB address}\\
For GPIB devices, the GPIB address must be defined.\\
\textit{Example:}
\begin{lstlisting}
GPIB address = 3
\end{lstlisting}

{\bfseries GPIB converter}\\
This defines the GPIB interface converter (eg RS232 to GPIB).
Valid values are `Micro488'.\\
\textit{Example:}
\begin{lstlisting}
GPIB converter = Micro488
\end{lstlisting}

{\bfseries header generator}\\
A header for the log file can be optionally added to the log file, using the output
of a user provided script. Output should be to \cc{stdout}.
Each line will have a ``\#'' automatically prepended to it.\\
\textit{Example:}
\begin{lstlisting}
header generator = bin/myticheader.pl
\end{lstlisting}

{\bfseries lock file}\\
This defines the device lock file, used to prevent multiple instances of the logger
from being started.\\
\textit{Example:}
\begin{lstlisting}
lockfile = okxem.gpscv.lock
\end{lstlisting}

{\bfseries logger}\\
This defines the counter logging script.\\
\textit{Example:}
\begin{lstlisting}
logger = okxemlog.pl
\end{lstlisting}

{\bfseries logger options}\\
This defines options passed to the counter logging script.\\
\textit{Example:}
\begin{lstlisting}
logger options =
\end{lstlisting}

{\bfseries okxem channel}\\
The OpenTTP counter is multi-channel so the channel to use (1-6) must be specified.\\
\textit{Example:}
\begin{lstlisting}
okxem channel=3
\end{lstlisting}

{\bfseries port}\\
This defines the port used to communicate with the counter. It's value depends on the type of counter. 
For the XEM6001, it's a Unix socket. For serial devices, it's a device name like
\cc{/dev/ttyUSB0}.\\
\textit{Example:}
\begin{lstlisting}
# this is the port used by okcounterd
port = 21577 
\end{lstlisting}

{\bfseries flip sign}\\
The processing software eg \cc{mktimetx} assumes that TIC measurements are started by the reference
and stopped by the GNSS receiver. If you need to invert the sign of the measurements, set the option
`flip sign' to `yes',
\textit{Example:}
\begin{lstlisting}
flip sign = no 
\end{lstlisting}

\subsection{[Misc] section}

\hypertarget{h:misc}{}

{\bfseries gzip}\\
Defines the compression/decompression program used in conjunction with counter and receiver log files.\\
\textit{Example:}
\begin{lstlisting}
gzip = /bin/gzip 
\end{lstlisting}

\subsection{[Delays] section}

\hypertarget{h:delays}{}

{\bfseries antenna cable}\\
This is ANT DLY as used in the CGGTTS header. Units are ns.\\
\textit{Example:}
\begin{lstlisting}
antenna cable = 0.0
\end{lstlisting}

{\bfseries reference cable}\\
This is REF DLY as used in the CGGTTS header. Units are ns.\\
\textit{Example:}
\begin{lstlisting}
reference cable = 0.0
\end{lstlisting}

\subsection{[Paths] section} 

\hypertarget{h:paths}{}

Paths follow the rules described in .

{\bfseries root}\\ \hypertarget{h:rootpath}{}
Defines the root path to be used with all other paths, unless they are specified as absolute paths.
As with other paths specified in \cc{gpscv.conf}, it is interpreted as being relative to the user's
home directory, unless prefaced with a `/'.\\
\textit{Example:}
\begin{lstlisting}
root = test/newreceiver
\end{lstlisting}

{\bfseries CGGTTS}\\
Defines the default directory used for CGGTTS files.
This is typically overridden by the output directory that can be  defined in each CGGGTS output section.\\
\textit{Example:}
\begin{lstlisting}
CGGTTS = cggtts
\end{lstlisting}

{\bfseries counter data}\\
Defines the directory used for TIC data files.\\
\textit{Example:}
\begin{lstlisting}
counter data = raw
\end{lstlisting}

{\bfseries processing log}\\
Defines the directory where the \cc{mktimetx} processing log is written.\\
\textit{Example:}
\begin{lstlisting}
processing log = logs
\end{lstlisting}

{\bfseries receiver data}\\
Defines the directory used for GNSS receiver raw data files.\\
\textit{Example:}
\begin{lstlisting}
receiver data = raw
\end{lstlisting}

{\bfseries RINEX}\\
Defines the directory used for RINEX files.\\
\textit{Example:}
\begin{lstlisting}
RINEX = rinex
\end{lstlisting}

{\bfseries tmp}\\
Defines the directory used for intermediate and debugging files.\\
\textit{Example:}
\begin{lstlisting}
tmp = tmp
\end{lstlisting}

{\bfseries onewire temp data}\\
Defines the directory used to write temperature logs.\\
\textit{Example:}
\begin{lstlisting}
onewiretemp data = raw/onewire
\end{lstlisting}

{\bfseries uucp lock}\\
Sets the directory used to write UUCP lock files. UUCP lock files are used with serial devices to prevent
other processes (eg \cc{minicom}) opening the serial port while it is in use. The default is \cc{/var/lock}
but this varies with the operating system and its version so you will need to check this.\\
\textit{Example:}
\begin{lstlisting}
# This is for Debian
uucp lock = /var/run/lock
\end{lstlisting}

{\bfseries rinex l1l2}\\ \hypertarget{h:rinex_l1l2}{}
Defines the directory used to write RINEX files with all observations, typically so that precise antenna coordinates
can be obtained. This is only used with dual frequency Javad receivers.\\
\textit{Example:}
\begin{lstlisting}
rinex l1l2 = rinex/l1l2
\end{lstlisting}


\subsection{[Receiver] section \label{sgcreceiver}}

\hypertarget{h:receiver}{}

{\bfseries configuration}\\
\\
\textit{Example:}
\begin{lstlisting}
configuration = etc/rx.conf
\end{lstlisting}

{\bfseries elevation mask}\\
This sets an elevation mask for tracking os satellites - below this elevation, satellites
are ignored. The units are degrees. This may not be implemented for all receivers.\\
\textit{Example:}
\begin{lstlisting}
elevation mask = 0
\end{lstlisting}

{\bfseries logger}\\
This is the script used to configure and log messages from the GNSS receiver.\\
\textit{Example:}
\begin{lstlisting}
logger = jnslog.pl
\end{lstlisting}

{\bfseries logger options}\\
These are options passed to the receiver logging script.\\
\textit{Example:}
\begin{lstlisting}
logger options =
\end{lstlisting}

{\bfseries manufacturer}\\
This defines the manufacturer of the receiver. Together with the model and version, 
this sets how data from the receiver is processed. For a list of supported receivers
see XX.\\
\textit{Example:}
\begin{lstlisting}
manufacturer = Javad
\end{lstlisting}

{\bfseries model}\\
This is the receiver model.For a list of supported receivers
see XX.\\
\textit{Example:}
\begin{lstlisting}
model = HE_GGD
\end{lstlisting}

{\bfseries version}\\
This can be used to identify the firmware version in use, for example.\\
\textit{Example:}
\begin{lstlisting}
version = 2.6.1
\end{lstlisting}

{\bfseries observations}\\
This is a list of GNSS systems which will be tracked by the receiver. 
In some applications where a multi-GNSS receiver is used, it may be desirable to track
only one GNSS system so this sets which system is used.
More generally, low-end multi-GNSS receivers are typically only capable of tracking
certain combinations of GNSS systems so this is used to select the required combination.
\textit{Example:}
\begin{lstlisting}
observations = GPS
\end{lstlisting}

{\bfseries port}\\
This is the serial port used for communication with the receiver.\\
\textit{Example:}
\begin{lstlisting}
port = /dev/ttyS0
\end{lstlisting}

{\bfseries pps offset}\\ 
This is an offset programmed into the GNSS receiver. Its purpose is to ensure that the counter
triggers correctly, particularly HP5313x counters, which will only trigger every two seconds if the 
start trigger slips slightly behind the stop trigger. Suitable values are determined by the long-term stability of the
reference, compared with GPS.
The units are ns.\\
\textit{Example:}
\begin{lstlisting}
pps offset = 3500
\end{lstlisting}

{\bfseries configuration}\\ \hypertarget{h:configuration}{}
This specfies a file to be used to configure the receiver. Currently, it is only used
with Javad receivers.\\
\textit{Example:}
\begin{lstlisting}
configuration = rx.cfg
\end{lstlisting}

{\bfseries pps synchronization}\\ \hypertarget{h:pps_synchronization}{}
This is a Javad-specific option. The logging script will force a synchronization of the receiver's
internal time scale with the input 1 pps.\\
\textit{Example:}
\begin{lstlisting}
pps synchronization = no
\end{lstlisting}

{\bfseries pps synchronization delay}\\ \hypertarget{h:pps_synchronization_delay}{}
This is a Javad-specific option. Synchronization of the receiver's
internal time scale with the input 1 pps is delayed for this time after reset of the receiver.
The units are seconds.\\
\textit{Example:}
\begin{lstlisting}
pps synchronization delay = 300
\end{lstlisting}

{\bfseries status file}\\
The status file contains a summary of the current state of the receiver,tyically at least the
currently visible GNSS satellites. Other software, for example \cc{lcdmonitor}, uses this information.\\
\textit{Example:}
\begin{lstlisting}
status file = logs/rx.status
\end{lstlisting}

{\bfseries timeout}\\
The logging script will time out and exit if no messages are received for this period.\\
\textit{Example:}
\begin{lstlisting}
timeout = 60
\end{lstlisting}

{\bfseries sawtooth phase}\\
This defines which pps the sawtooth correction is to be applied to.
Valid values are `current second', `next second' and `receiver specified'.
`Current' means for the pps just generated.\\
\textit{Example:}
\begin{lstlisting}
sawtooth phase = current second
\end{lstlisting}

{\bfseries year commissioned}\\
The year the receiver was commissioned. This is used in the CGGTTS header.\\
\textit{Example:}
\begin{lstlisting}
year commissioned = 1999
\end{lstlisting}

\subsection{[Reference] section}

\hypertarget{h:reference}{}

{\bfseries file extension}\\
This defines the extension used for Reference status logs.\\
\textit{Example:}
\begin{lstlisting}
file extension = .rb
\end{lstlisting}

{\bfseries logging interval}\\
This defines the interval between status file updates. The units are seconds.\\
\textit{Example:}
\begin{lstlisting}
logging interval = 60
\end{lstlisting}

{\bfseries log path}\\
This defines where status logs are written to.\\
\textit{Example:}
\begin{lstlisting}
log path = raw
\end{lstlisting}

{\bfseries log status}\\
This enables status logging of the Reference.\\
\textit{Example:}
\begin{lstlisting}
log status = yes
\end{lstlisting}

{\bfseries oscillator}\\
This identifies the installed oscillator, so that device-specific handling can be implemented.\\
\textit{Example:}
\begin{lstlisting}
oscillator = PRS10
\end{lstlisting}

{\bfseries power flag}\\
This defines the file used to flag that the Reference has lost power, and needs rephasing.
Currently, this only has meaning for the PRS10. It is used ntpd to disable the refclock
corresponding to the PRs10's 1 pps.\\
\textit{Example:}
\begin{lstlisting}
power flag = logs/prs10.pwr
\end{lstlisting}

{\bfseries status file}\\
In the case of the PRS10, this consists of the six status bytes and sixteen ADC values.\\
\textit{Example:}
\begin{lstlisting}
status file = logs/prs10.status
\end{lstlisting}


\subsection{[RINEX] section}

\hypertarget{h:rinex}{Entries} in this section control the format and content of RINEX files.
RINEX observation and navigation files in version 2 and version 3 formats can be produced.
RINEX observations 

{\bfseries agency}\\
This specifies the value of the AGENCY field which appears in RINEX observation file headers.\\
\textit{Example:}
\begin{lstlisting}
agency = MY AGENCY
\end{lstlisting}

{\bfseries create}\\
This defines whether or not RINEX files will be generated.\\
\textit{Example:}
\begin{lstlisting}
create = yes
\end{lstlisting}

{\bfseries force v2 name}\\
This forces a V2 name for V3 RINEX output.\\
\textit{Example:}
\begin{lstlisting}
force v2 name = no
\end{lstlisting}

{\bfseries observer}\\
Normally, only code observations are output by mktimetx. 
To output phase observations, set this to 'all'.\\
\textit{Example:}
\begin{lstlisting}
observations = code
\end{lstlisting}

{\bfseries observer}\\
This specifies the value of the OBSERVER field which appears in RINEX observation file headers.
If the observer is specified as `user' then the environment variable USER is used.\\
\textit{Example:}
\begin{lstlisting}
observer = user
\end{lstlisting}

{\bfseries version}\\
This  specifies the version of the RINEX output. Valid versions are 2 and 3.\\
\textit{Example:}
\begin{lstlisting}
version = 2
\end{lstlisting}


