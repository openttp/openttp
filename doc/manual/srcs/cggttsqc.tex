\subsection{cggttsqc.py}

\hypertarget{h:cggttsqc}{}

\cc{cggttsqc.py} performs various checks on CGGTTS files.
Running it on a single CGGTTS file with no options will produce output like:

\begin{tabular}{lrrrrrr}
File         & Tracks  & Short & Min SV & Max SV   &  High DSG  &  Low ELV \\
57401.cctf   &   706   &  71   &   3    &  11      & 0     &  0   \\
\end{tabular}

where:\\
`Tracks' is the total number of tracks\\
`Short'  is the number of tracks with length less than a specified threshold (default 780 s)\\
`Min SV' is the minimum number of SVs visible\\
`Max SV' is the maximum number of SVs visible \\
`High DSG'    is the number of tracks with DSG above a specified threshold (default 20 ns)\\
`Low ELV'    is the number of tracks with elevation below a specified threshold (default 10 degrees)\\

\subsubsection{usage}

\begin{lstlisting}[mathescape=true]
cggttsqc.py [OPTION] $\ldots$ file [file $ldots$]
\end{lstlisting}
The command line options are:
\begin{description*}
	\item[-{}-help,-h]	print the help information and exit
	\item[-{}-debug,-d]	print debugging information to stderr
	\item[-{}-nowarn]   suppress warnings (eg about missing files, bad formatting in CGGTTS files)
	\item[-{}-dsg \textless{value}\textgreater]  set the upper limit for acceptable DSG. The units are ns.
	\item[-{}-elevation \textless{value}\textgreater] set the lower limit for acceptable elevation. The units are degrees.
	\item[-{}-tracklength  \textless{value}\textgreater] set the lower limit for acceptable track length. The units are seconds.
	\item[-{}-checkheader] show when significant fields in the header change, for example the delays.
	\item[-{}-nosequence] do not interpret (two) input filenames as a sequence.
	\item[-{}-plotcount]  plot a histogram of satellite count at each scheduled track time
	\item[-{}-version,-v] show the version information and exit
\end{description*}

\subsubsection{examples}

Mutiple files can be checked and a sequence can be specified with two file names.
For example:
\begin{lstlisting}
cggttsqc.py --dsg 5 GZAA0157.834 GZAA0157.876
\end{lstlisting}
will report on all files between MJDs 57834 and 57876, indicating in the DSG field the number of tracks with DSG greater
than 5 ns. For the two file names to be interpreted as a sequence, they must:
\begin{enumerate}
	\item have names in v2E CGGTTS recommended format or in the format MJD.xxx
	\item be in the same directory
	\item have the same extension, if in the format MJD.xxx
\end{enumerate}
