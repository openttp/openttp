


This chapter gives a short introduction to the Open Traceable Time Platform (OpenTTP), describing the Reference Platform and other supported hardware.

\section{What is OpenTTP?}

The Open Traceable Time Platform is an open-source solution for a timing system that can be made fully traceable to national standards.
It achieves traceability using the GPS common-view technique, which allows distant clocks to be compared with an accuracy of a few ns.
The reference platform is based on readily available, low-cost OEM modules and provides a full software and hardware solution. 

The goals of the OpenTTP project are:
\begin{enumerate}

	\item Fully open source hardware and software 
	
	\item Easy customisation for specialised applications
	
	\item Production of time-transfer files in the standard CGGTTS data format
	
	\item Easy extension to new receivers

	\item Low cost
	
	\item Provision of a convenient framework for further research and development.

\end{enumerate}

Applications currently include provision of traceable time of day and auditing of NTP-synchronized systems.

\section{The OpenTTP software suite}

The OpenTTP software suite provides a full solution for automated logging and processing of time-transfer data.
It is available via GitHub:
\begin{lstlisting}
https://github.com/openttp
\end{lstlisting}
and users are invited to contribute to its development.

The software currently supports a number of GNSS receivers and counters. 
The supported GNSS receivers are mostly low-cost, single-frequency receivers since low cost is a key objective of 
the OpenTTP project.
	
	\subsection{Supported GNSS receivers}
	
	OpenTTP currently supports the receivers listed in \ref{t:receivers}
	
	\begin{table}[h]
	\begin{tabular}{lll}
	Manufacturer & models & notes \\ \hline
	Javad & GRIL receivers & obsolete \\
	NVS   & NV08 & \\
	Trimble & Resolution T & obsolete\\
	ublox & NEO-M8T,ZED-F9P,ZED-F9T & may work with earlier receivers\\
	\end{tabular}
	\caption{Supported GPS/GNSS receivers.\label{t:receivers}}
	\end{table}
	
	OpenTTP uses a custom file format for logging GPS receiver data. 
	It does not read native receiver binary-format files.
	
	Guidance on testing a receiver for suitability for time-transfer, and writing software to process
	the receiver's data, is given in the OpenTTP Developer's Guide.
	
	\subsection{Supported counters}
	
	Counters supported by OpenTTP are listed in \ref{t:counters}.
	
	\begin{table}[h]
	\begin{tabular}{lll}
	Manufacturer & models & notes \\ \hline
	Agilent & 5313x &  needs IOTech GPIB to RS232 converter\\
	OpenTTP & XEM6001 & simple FPGA-based counter/timer \\
	SRS & PRS10 & uses the input 1 pps time-tagging function\\
	TAPR & TICC &\\
	\end{tabular}
	\caption{Supported counters. \label{t:counters}}
	\end{table}
	
	The file format used \ref{s:TICformat} is very simple and it should be easy to convert from another format, if needed.
	
\section{The OpenTTP reference platform}

The OpenTTP reference platform consists of:
\begin{itemize}
\item BeagleBone Black, an ARM-based single board computer
\item NVS Technologies NV08-CSM GNSS receiver
\item Opal Kelly XEM6001 FPGA development board
\item Jackson Labs LTE Lite GPS-disciplined oscillator
\item Solid-state disk for mass storage
\item CrystalFontz LCD module
\end{itemize}

Custom circuits, printed circuit board designs and other hardware resources are all available via the GitHub repository.

\subsection{Licenses}

The software is available under the MIT license.
