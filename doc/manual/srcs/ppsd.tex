\section{ppsd \label{s:ppsd}}

\cc{ppsd} produces a digital pulse on an I/O port, aligned with the system time.
It works by sleeping until just before the second (plus any programmed delay) rolls over, 
and then going into a hard loop, polling the time until rollover.

Two I/O ports are presently supported: the standard PC parallel port, and SIO8186x devices.
The latter are found on some single board computers. For parallel port output, all 8 bits 
of the data port (pins 2 to 9 on a DB37) are written to.

\cc{ppsd} doesn't produce a log file.

\subsection{usage}

To run \cc{ppsd} on the command line, use:
\begin{lstlisting}[mathescape=true]
ppsd [option] $\ldots$
\end{lstlisting}
The command line options are:
\begin{description*}
	\item[-d]	run in debugging mode
	\item[-h]	print help and exit
	\item[-o \textless delay\textgreater] set the PPS delay, in microseconds.
	\item[-v]	print version information and exit
\end{description*}

\subsection{configuration file}

The configuration file, \cc{ppsd.conf} contains a single number, an offset for the output 1 pps, in microseconds.

