\section{rnx2cggtts \label{s:rnx2cggtts}}

\cc{rnx2cggtts} generates CGGTTS files from RINEX observation and navigation files.
Currently, it will only generate CGGTTS from GPS measurements. RINEX files must be version 3.

CGGTTS outputs have been compared with the output of \cc{r2cggtts}.
Known differences are:
\begin{enumerate}
\item Occasionally, a different IODE will be chosen. When comparing CGGTTS between r2cggtts and rnx2cggtts, 
tracks should be matched on IODE, otherwise there will be outliers at the 5 to 10 ns level.
\item In P3 files, MDIO is modelled ionosphere, and not MSIO
\item Satellite ELV and AZM are evaluated from a fit to the trajectory, rather than the middle point of the data set for a track. 
\end{enumerate}

\subsection{usage}

To run \cc{rnx2cggtts.py} on the command line, use:
\begin{lstlisting}[mathescape=true]
rnx2cggtts [option] $\ldots$
\end{lstlisting}
The command line options are:
\begin{description*}
	\item[--configuration FILE, -c  FILE] use the specified configuration file
	\item[--debug FILE, -d FILE]	print debugging information to file-
	\item[--shorten] shorten debugging messages
	\item[--verbosity \textless{n}\textgreater ]  set debugging verbosity
	\item[--help, -h]	show help and exit
	\item[--licence] show the software licence and exit
	\item[--version, -v]	print version information and exit
\end{description*}

\subsection{configuration file}

The configuration file is similar to \cc{gpscv.conf}.
A minimal configuration file for P3 output is:

\begin{lstlisting}
[RINEX]
station = MOST01AUS

[Antenna]
X = -4648240.85 
Y = +2560636.45 
Z = -3526317.79 

[CGGTTS]
outputs = CGGTTS-GPS-P3

[GPS delays]
kind = internal
codes =  C1C,C1W,C2W
delays = 0.0,0.0,0.0
BIPM cal id = UNCALIBRATED

[CGGTTS-GPS-P3]
constellation=GPS
code=C1W+C2W
path=cggtts

[Paths]
rinex observations = RINEX
rinex navigation   = RINEX

\end{lstlisting}

\subsubsection{[RINEX] section }

{\bfseries station}\\
This the RINEX station name. 
Both V2 (4 characters) and V3 (9 characters) are allowed.
The station name is used to construct RINEX file names.
Currently, these files are expected to be decompressed.\\
\textit{Example:}
\begin{lstlisting}
station = MOST01AUS
\end{lstlisting}

\subsubsection{[Antenna] section }

{\bfseries X,Y,Z}\\
These are the ECEF antenna coordinates, in metres.
They are used in the calculations and are written to the CGGTTS header.\\
\textit{Example:}
\begin{lstlisting}
X = -4648240.85 
Y = +2560636.45 
Z = -3526317.79
\end{lstlisting}


{\bfseries frame}\\
This is the reference frame applicable to the antenna coordinates.
It is only used in the CGGTTS header and is optional.\\
\textit{Example:}
\begin{lstlisting}
frame = ITRF2014
\end{lstlisting}
	
\subsubsection{[Receiver] section }

{\bfseries manufacturer}\\
This is identifies the manufacturer of the GNSS receiver.
It is only used in the CGGTTS header in the RCVR line and is optional.\\
\textit{Example:}
\begin{lstlisting}
manufacturer = Septentrio
\end{lstlisting}


{\bfseries model}\\
This identifies the GNSS receiver model.
It is only used in the CGGTTS header in the RCVR line and is optional.\\
\textit{Example:}
\begin{lstlisting}
model = mosaic-T
\end{lstlisting}

{\bfseries serial number}\\
This identifies the GNSS receiver serial number.
It is only used in the CGGTTS header in the RCVR line and is optional.\\
\textit{Example:}
\begin{lstlisting}
serial number = 123456
\end{lstlisting}

{\bfseries commissioning year}\\
This identifies year the GNSS receiver started operation.
It is only used in the CGGTTS header in the RCVR line  and is optional.\\
\textit{Example:}
\begin{lstlisting}
commissioning year = 2022
\end{lstlisting}

{\bfseries channels}\\
This identifies the number of receiver channels.
It is only used in the CGGTTS header and is optional.\\
\textit{Example:}
\begin{lstlisting}
channels = 768
\end{lstlisting}

\subsubsection{[CGGTTS] section }

{\bfseries outputs}\\
This is a comma-separated list of CGGTTS output sections.
Each item in the list must have a corresponding section.
\\
\textit{Example:}
\begin{lstlisting}
outputs = CGGTTS-GPS-P3
\end{lstlisting}

{\bfseries version}\\
This is the CGGTTS version. 
Only V2E is supported at present.
The version is optional and the default is V2E.
\\
\textit{Example:}
\begin{lstlisting}
version = V2E
\end{lstlisting}

{\bfseries naming convention}\\
This specifies the style of the CGGTTS file name.
\cc{Plain} specifies a filename in the format \cc{MJD.cctf}.
\cc{BIPM} specifies a filename in the format used by the BIPM.
This is optional and the default is \cc{Plain}.\\
\textit{Example:}
\begin{lstlisting}
naming convention = BIPM
\end{lstlisting}


{\bfseries reference}\\
This specifies the laboratory reference.
This is used in the CGGTTS header and is optional; the default is \cc{UTC(XLAB)}.\\
\textit{Example:}
\begin{lstlisting}
reference = UTC(AUS)
\end{lstlisting}

{\bfseries lab}\\
This specifies the laboratory.
This is used in the CGGTTS header and is optional;the default is \cc{XLAB}.\\
\textit{Example:}
\begin{lstlisting}
lab = NMIA
\end{lstlisting}

{\bfseries comments}\\
This is used in the CGGTTS header and is optional;the default is an empty string.
\textit{Example:}
\begin{lstlisting}
comments = antenna moved MJD 59601
\end{lstlisting}

{\bfseries ref dly}\\
This is the reference delay (REF DLY).
Its units are nanoseconds.
Its use is optional and the default value is zero.
\textit{Example:}
\begin{lstlisting}
ref dly = 99.3
\end{lstlisting}

{\bfseries cab dly}\\
This is the cable delay (CAB DLY).
Its units are nanoseconds.
Its use is optional and the default value is zero.
\textit{Example:}
\begin{lstlisting}
cab dly = 120.2
\end{lstlisting}

{\bfseries minimum track length}\\
This sets the minimum track length that is acceptable for reporting a satellite track.
The units are seconds.
This is optional and the default is 390 s.\\
\textit{Example:}
\begin{lstlisting}
minimum track length = 750
\end{lstlisting}

{\bfseries maximum dsg}\\
This sets the maximum value of DSG that is acceptable for reporting a satellite track.
The units are nanoseconds.
This is optional and the default is 100 ns.\\
\textit{Example:}
\begin{lstlisting}
maximum dsg = 10
\end{lstlisting}

{\bfseries minimum elevation}\\
This sets the minimum satellite elevation (at the middle of the track) that is acceptable for reporting a satellite track.
The units for this option are degrees.
This is optional and the default is 10 degrees.\\
\textit{Example:}
\begin{lstlisting}
minimum elevation = 15
\end{lstlisting}

{\bfseries maximum ura}\\
This sets the maximum URA for a broadcast ephemeris to be acceptable for use in calculations.
Typically, a receiver will report 2 metres, with a few reported at 2.8 metres.
IGS ephemerides however  will sometimes contain entries with very high URA and these should not be used.
The units for this option are metres.
This is optional and the default is 3 metres.
\textit{Example:}
\begin{lstlisting}
maximum ura = 3.0
\end{lstlisting}
	
\subsubsection{[GPS delays] section }


{\bfseries kind}\\
This specifies the kind of delay for the specified . Valid values are \cc{internal},\cc{system} and \cc{total}.\\
\textit{Example:}
\begin{lstlisting}
kind = internal
\end{lstlisting}

{\bfseries codes}\\
This defines a comma-separated list of codes for which delays are defined.
Use RINEX V3 pseudorange observation codes, as per page 17 of the RINEX V3.04 specification.\\
\textit{Example:}
\begin{lstlisting}
codes =  C1C,C1W,C2W
\end{lstlisting}

{\bfseries delays}\\
This sets the value of each delay defined by \cc{codes}, given as a comma-separated list in the same order as in \cc{codes}. The units are ns.  \\
\textit{Example:}
\begin{lstlisting}
delays = 23.0,22.0,24.0
\end{lstlisting}

{\bfseries BIPM cal id}\\
This sets the BIPM calibration identifier, written to the CGGTTS header.
It overrides the global [CGGTTS] option of the same name.
Its use is optional.\\
\textit{Example:}
\begin{lstlisting}
BIPM cal id = UNCALIBRATED
\end{lstlisting}

\subsubsection{[Paths] section }

{\bfseries rinex observations}\\
The specifies the path to the RINEX observation files.\\
\textit{Example:}
\begin{lstlisting}
rinex observations = RINEX
\end{lstlisting}

{\bfseries rinex navigation}\\
The specifies the path to the RINEX navigation files.\\
\textit{Example:}
\begin{lstlisting}
rinex navigation = RINEX
\end{lstlisting}

\subsubsection{CGGTTS output sections }

{\bfseries constellation}\\
Valid values are GPS, Galileo, GLONASS and Beidou.
Only GPS is supported at present though.\\
\textit{Example:}
\begin{lstlisting}
constellation = GPS
\end{lstlisting}

{\bfseries code}\\
This specifies the code combination to use in the CGGTTS output.
For ionosphere-free combinations, both codes are specified, separated by a '+'.
Use RINEXV3 pseudorange observation codes, as per page 17 of the RINEX V3.04 specification.
These need to be the same as used in the RINEX observation file.\\
\textit{Example:}
\begin{lstlisting}
code = C1W + C2W
\end{lstlisting}


{\bfseries path}\\
This specifies the path that the files will be written to.\\
\textit{Example:}
\begin{lstlisting}
path = cggtts\C1
\end{lstlisting}

{\bfseries report msio}\\
For single frequency outputs, this indicates whether or not to use MSIO.\\
\textit{Example:}
\begin{lstlisting}
report msio = yes
\end{lstlisting}

{\bfseries msio codes}\\
For single frequency outputs where use of MSIO is enabled, this specifies the codes to use for calculating MSIO.
Use RINEX V3 pseudorange observation codes, as per page 17 of the RINEX V3.04 specification.\\
\textit{Example:}
\begin{lstlisting}
msio codes = C1W + C2W
\end{lstlisting}


