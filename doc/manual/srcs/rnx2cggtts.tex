\section{rnx2cggtts \label{s:rnx2cggtts}}

\cc{rnx2cggtts} generates CGGTTS files from RINEX observation and navigation files.
Currently, it will only generate CGGTTS from GPS measurements. RINEX files must be version 3.

CGGTTTS outputs have been compared with the output of \cc{r2cggtts}.
Known differences are:
\begin{enumerate}
\item Occasionally, a different IODE will be chosen. When comparing CGGTTS between r2cggtts and rnx2cggtts, 
tracks should be matched on IODE, otherwise there will be outliers at the 5 to 10 ns level.
\item In P3 files, MDIO is modelled ionosphere, and not MSIO
\item Satellite ELV and AZM are evaluated from a fit to the trajectory, rather than the middle point from the data set for a track. 
\end{enumerate}

\subsection{usage}

To run \cc{kickstart.py} on the command line, use:
\begin{lstlisting}[mathescape=true]
rnx2cggtts [option] $\ldots$
\end{lstlisting}
The command line options are:
\begin{description*}
	\item[--configuration FILE, -c  FILE] use the specified configuration file
	\item[--debug FILE, -d FILE]	print debugging infomration to file-
	\item[--shorten] shorten debugging messages
	\item[--verbosity <n>]  set debugging verbosity
	\item[--help, -h]	show help and exit
	\item[--licence] show the software licence and exit
	\item[--version, -v]	print version information and exit
\end{description*}

\subsection{configuration file}

The configuration file is similar to \cc{gpscv.conf}.
A minimal configuration file for P3 output is:

\begin{lstlisting}
[RINEX]
station = MOST01AUS

[Antenna]
X = -4648240.85 
Y = +2560636.45 
Z = -3526317.79 

[CGGTTS]
outputs = CGGTTS-GPS-P3

[GPS delays]
kind = internal
codes =  C1C,C1W,C2W
delays = 0.0,0.0,0.0
BIPM cal id = UNCALIBRATED

[CGGTTS-GPS-P3]
constellation=GPS
code=C1W+C2W
path=cggtts

[Paths]
rinex observations = RINEX
rinex navigation   = RINEX

\end{lstlisting}

\subsubsection{[RINEX] section }

\subsubsection{[Antenna] section }

\subsubsection{[CGGTTS] section }

\subsubsection{[GPS delays] section }

\subsubsection{[Paths] section }



