\section{kickstart.py \label{s:kickstart}}

\cc{kickstart.py} is used to check that required processes are running,
and restart them if necessary. 
It is used to start the receiver and counter logging processes, for example.

The lock file for each process (target) contains the running process ID; 
this is used by \cc{kickstart.py} to test whether the target is running.

\cc{kickstart.py} produces a log file in the user's home directory, \cc{logs/kickstart.log}. Each time 
a process is checked, it touches the file \cc{logs/kickstart.target.check}, where \cc{target} is
specified in the configuration file.

\subsection{usage}

To run \cc{kickstart.py} on the command line, use:
\begin{lstlisting}[mathescape=true]
kickstart.py [option] $\ldots$
\end{lstlisting}
The command line options are:
\begin{description*}
	\item[--config FILE, -c  FILE] use the specified configuration file
	\item[--debug, -d ]	run in debugging mode
	\item[--help, -h]	print help and exit
	\item[--version, -v]	print version information and exit
\end{description*}

\subsection{configuration file}

An example configuration file is:
\begin{lstlisting}
targets = okxem
[okxem]
target = okxem
command = bin/okxemlog.pl 
lock file = logs/okxem.gpscv.lock
\end{lstlisting}

{\bfseries targets}\\
This defines a list of sections, each of which corresponds to a process to monitor.\\
\textit{Example:}
\begin{lstlisting}
targets = restlog,okxem
\end{lstlisting}

{\bfseries target}\\
This defines a target identifier. It is used to construct filenames and as an identifier in logged messages.\\
\textit{Example:}
\begin{lstlisting}
target = okxem
\end{lstlisting}

{\bfseries command}\\
This defines the command used to start the target. Options can be used.\\
\textit{Example:}
\begin{lstlisting}
command = bin/okxemlog.pl 
\end{lstlisting}

{\bfseries lock file}\\
This defines the lock file associated with a target.\\
\textit{Example:}
\begin{lstlisting}
lock file = logs/okxem.gpscv.lock
\end{lstlisting}

Paths specified in the configuration file are constructed using the usual rules.


